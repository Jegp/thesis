The following code simulates the data and produces the plots in Section \ref{sec:verification}.

\section{Single-neuron population rate experiments}
\begin{lstlisting}[language=Python]
import volrpynn.nest as v
import numpy as np
import pyNN.nest as pynn
from sklearn.linear_model import LinearRegression
from sklearn.metrics import r2_score
import matplotlib.pyplot as plt
get_ipython().magic('matplotlib inline')

parameters = {"tau_syn_I":5,"tau_refrac":0,"v_thresh":-50,"v_rest":-65,"tau_syn_E":5,"v_reset":-65,"tau_m":20,"e_rev_I":-70,"i_offset":0,"cm":1,"e_rev_E":0}
pynn.setup()

# Setup initial population
p1 = pynn.Population(1, pynn.IF_cond_exp(**parameters))
p1.record(['spikes', 'v'])

def simulate(offset):
    for recorder in pynn.simulator.state.recorders:
        recorder.clear()
    pynn.reset()
    p1.set(i_offset=offset)
    pynn.run(50)
    return p1.get_data()

def membrane_simulate(offset, pop):
    simulate(offset)
    b = pop.get_data()
    return b.segments[0].filter(name='v')[0]
    
def plot_membrane_simulate(offset, pop):
    current = membrane_simulate(offset, pop)
    spikes = len(pop.get_data().segments[0].spiketrains[0])
    plt.gca().set_title('Spikes: ' + str(spikes))
    plt.plot(np.arange(0, 50.1, 0.1), current)
    
def spikes_simulate(offset, pop):
    simulate(offset)
    b = pop.get_data()
    return len(b.segments[0].spiketrains[0])

# Membrane current plot
plot_membrane_simulate(2, p1)
plt.gca().set_title('')
plt.gcf().set_size_inches(6, 4)
plt.gca().set_xlabel('Simulation time in ms')
plt.gca().set_ylabel('Membrane potential in mV')
plt.savefig('membrane.svg')

# Spike rate regression model
xs = np.arange(0, 12.6, 0.02)
spikes = [spikes_simulate(x, p1) for x in xs]
reg = LinearRegression().fit(xs.reshape(-1, 1), spikes)
print(reg.coef_, reg.intercept_)
pred_y = reg.predict(xs.reshape(-1, 1))
r2_score(spikes, pred_y)

# Plot spike count and rate for first population
plt.gca().plot(xs, spikes)
plt.gca().set_ylabel('Number of generated spikes')
plt.gca().set_xlabel('Constant input current in nA')
plt.gca().set_title('')
plt.gcf().set_size_inches(6, 4)
plt.plot(xs, pred_y, color='black', linewidth=0.6, label="f(x) = 3.225x - 1.615", linestyle="-.")
plt.legend()
ylim1, ylim2 = plt.gca().get_ylim()
ax2 = plt.gca().twinx()
ax2.set_ylim(ylim1 / 50, ylim2 / 50)
ax2.set_ylabel('Spike rate (N = 50 ms)')
plt.savefig('spike_rate.svg')


# Deeper layer
p2 = pynn.Population(1, pynn.IF_cond_exp(**parameters))
proj2 = pynn.Projection(p1, p2, pynn.AllToAllConnector())
p3 = pynn.Population(1, pynn.IF_cond_exp(**parameters))
proj3 = pynn.Projection(p2, p3, pynn.AllToAllConnector())
p4 = pynn.Population(1, pynn.IF_cond_exp(**parameters))
proj4 = pynn.Projection(p3, p4, pynn.AllToAllConnector())
p2.record(['v', 'spikes'])
p3.record(['v', 'spikes'])
p4.record(['v', 'spikes'])

def spikes_simulate_deep(offset, weight_function):
    proj2.set(weight=weight_function(offset, p1.size, p2.size))
    proj3.set(weight=weight_function(offset, p2.size, p3.size))
    proj4.set(weight=weight_function(offset, p3.size, p4.size))
    simulate(offset)
    return (p1.get_data(), p2.get_data(), p3.get_data(), p4.get_data())

def to_spikes(d):
    return d.segments[0].spiketrains[0].size

def to_potential(d):
    return d.filter('v')

# Plot spike counts with constant weights
data_constant = [spikes_simulate_deep(x, lambda r, x, y: 1) for x in xs]
spikes_constant = np.array([(to_spikes(x[0]), to_spikes(x[1]), to_spikes(x[2]), to_spikes(x[3])) for x in data_constant])

plt.figure(figsize=(15, 4))
ax = plt.subplot(131)
ax.set_ylim(0, 500)
ax.set_title('Second population (N = 1)')
plt.ylabel('Number of generated spikes')
plt.xlabel('Constant input current in nA')
plt.plot(spikes_constant[:, 0], spikes_constant[:, 1])
ax2 = plt.subplot(132)
ax.set_ylim(0, 500)
ax2.set_title('Third population (N = 1)')
plt.ylabel('Number of generated spikes')
plt.xlabel('Number of input spikes')
plt.plot(spikes_constant[:, 1], spikes_constant[:, 2])
ax3 = plt.subplot(133)
ax.set_ylim(0, 500)
ax3.set_title('Fourth population (N = 1)')
plt.xlabel('Number of input spikes')
plt.plot(spikes_constant[:, 2], spikes_constant[:, 3])
plt.savefig('spike_rate_not_weighted.svg')

# Spike rates with adjusted weights
data = [spikes_simulate_deep(x, lambda r, x, y: 0.065 / x) for x in xs]
spikes = np.array([(to_spikes(x[0]), to_spikes(x[1]), to_spikes(x[2]), to_spikes(x[3])) for x in data])

plt.figure(figsize=(15, 4))
ax = plt.subplot(131)
ax.set_ylim(0, 40)
ax.set_title('Second population (N = 1)')
plt.ylabel('Number of generated spikes')
plt.xlabel('Number of input spikes')
plt.plot(spikes[:, 0], spikes[:, 1])
ax2 = plt.subplot(132)
ax2.set_ylim(0, 40)
ax2.set_title('Third population (N = 1)')
plt.ylabel('Number of generated spikes')
plt.xlabel('Number of input spikes')
plt.plot(spikes[:, 1], spikes[:, 2])
ax3 = plt.subplot(133)
ax3.set_ylim(0, 40)
ax3.set_title('Fourth population (N = 1)')
plt.xlabel('Number of input spikes')
plt.plot(spikes[:, 2], spikes[:, 3])
plt.savefig('spike_rate_chain.svg')
\end{lstlisting}

\section{MNIST neuron rate experiments}
\begin{lstlisting}[language=Python]
import volrpynn.nest as v
import numpy as np
import pyNN.nest as pynn
from sklearn.linear_model import LinearRegression
from sklearn.metrics import r2_score
import matplotlib.pyplot as plt
get_ipython().magic('matplotlib inline')
parameters = {"tau_syn_I":5,"tau_refrac":0,"v_thresh":-50,"v_rest":-65,"tau_syn_E":5,"v_reset":-65,"tau_m":20,"e_rev_I":-70,"i_offset":0,"cm":1,"e_rev_E":0}
pynn.setup()

# Setup populations and projections 

p1 = pynn.Population(100, pynn.IF_cond_exp(**parameters))
p2 = pynn.Population(100, pynn.IF_cond_exp(**parameters))
proj2 = pynn.Projection(p1, p2, pynn.AllToAllConnector())
p3 = pynn.Population(10, pynn.IF_cond_exp(**parameters))
proj3 = pynn.Projection(p2, p3, pynn.AllToAllConnector())
p1.record(['spikes', 'v'])
p2.record(['spikes', 'v'])
p3.record(['spikes', 'v'])

def simulate(offset):
    for recorder in pynn.simulator.state.recorders:
        recorder.clear()
    pynn.reset()
    p1.set(i_offset=offset)
    pynn.run(50)

def simulate_spikes(offset, weight_function):
    proj2.set(weight=weight_function(offset, p1.size, p2.size))
    proj3.set(weight=weight_function(offset, p2.size, p3.size))
    simulate(offset)
    return (p1.get_data(), p2.get_data(), p3.get_data())

def count_spikes(data):
    return np.array([s.size for s in data.segments[0].spiketrains]).mean()

def simulate_offsets(rates, weight_function):
    return np.array([simulate_spikes(rate, weight_function) for rate in rates])
    
# Define weight normalisation function
weight_function = lambda r, x, y: 0.065 / x
xs = np.arange(0, 12, 0.1)
data = simulate_offsets(ct, weight_function)

spikes = np.array([(count_spikes(d1), count_spikes(d2), count_spikes(d3)) for (d1, d2, d3) in data])

# Plot population rates

plt.figure(figsize=(15, 4))
ax = plt.subplot(131)
ax.set_ylim(0, 40)
ax.set_title('First population (N = 100)')
plt.ylabel('Number of generated spikes')
plt.xlabel('Constant input current in nA')
plt.plot(xs, spikes[:, 0])
ax2 = plt.subplot(132)
ax2.set_ylim(0, 40)
ax2.set_title('Second population (N = 100)')
plt.ylabel('Average number of generated spikes')
plt.xlabel('Average number of input spikes per neuron')
plt.plot(spikes[:, 0], spikes[:, 1])
ax3 = plt.subplot(133)
ax3.set_ylim(0, 40)
ax3.set_title('Third population (N = 10)')
plt.ylabel('Average number of generated spikes')
plt.xlabel('Average number of input spikes per neuron')
plt.plot(spikes[:, 1], spikes[:, 2])
plt.savefig('spike_rate_mnist.svg')

\end{lstlisting}
