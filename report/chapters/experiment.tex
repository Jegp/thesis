\documentclass[report.tex]{subfiles}
\begin{document}
This chapter describes the experimental setup that aims to validate the
implementation and test the hypotheses introduced in section \ref{sec:hypotheses}.
Firstly, the assumptions and parameters that are the basis of the simulations are described.
Second, datasets and methods that are used to test the hypotheses are
elaborated.

\section{Neuron parameters}
Since the neural network topologies between the backends are shown to be similar,
the parameters related to this topology are shared as well.
By applying the theory from \textcite{Rueckauer2017} (see section
\ref{sec:coding}), this section will explain how input data, network weights,
and network biases can be transferred from \glspl{ANN} to \glspl{SNN}.
It will then proceed to describe the setup with which 
%TODO KK finish above sentence

As explained in Section \ref{sec:coding}, normalised input in \glspl{ANN} can
be inserted into \glspl{SNN} with the help of a linear transformation.
This transformation has been examined empirically, by constructing a simple 
one-neuron network, and injecting a constant current over time.
By running a number of experiments, it is possible to measure the integrated
current in the neuron and the amount of spikes it produces over time.

\begin{figure}
  \centering
  \includegraphics[width=0.8\textwidth]{images/membrane.png}
  \caption{Membrane potential for a LIF neuron given a constant input current of
  2 nA, simulated over 100 ms.}
  \label{fig:membrane}
\end{figure}

Figure \ref{fig:membrane} shows one such experiment, in which the membrane
potential of a single neuron is plotted over time.
The neuron is given a constant input current of 1.3 nA, and 
produces 5 spikes during the 100 ms simulation time. 
As expected in section \ref{sec:coding}, the interspike interval is constant.

A neuron only spikes when its membrane potential exceeds its excitation threshold
$V_{thr}$, but depends on a number of parameters to describe the neuron
conductivity, current decay time etc.
In ll three, PyNN, NEST and BrainScaleS, these parameters can be programatically
defined. Table \ref{tab:neuron_parameters} shows the default parameters for
NEST (also shown in Listing \ref{lst:lif-cond-exp}).
The $\tau_{syn}$ parameters denote the decay time for the input spike currents. 
Similarly, the membrane time constant, $\tau_m$, expresses the time it takes for the
neuron membrane to decay to its resting state ($V_{rest}$) if no other
input arrives. 
For the LIF\index{neuron model!leaky-integrate-and-fire} model used in PyNN,
all $\tau$ parameters decay exponentially \cite{Davison2009}.
The $V_{rev}$ parameters explain the potential to integrate into the neuron,
when either excitatory or inhibitory input arrives. 

\renewcommand{\arraystretch}{1.3} 
\begin{table}
\begin{tabular}[center]{l r l l}
  \hline
  $C_m$ & 1 & nF & Capacity of the membrane. \\ \hline
  $I_{offset}$ & 0 & nA & Offset current. \\ \hline
  $V_{rest}$ & -65 & mV & Resting membrane potential. \\ \hline
  $V_{reset}$ & -65 & mV & Reset potential after a spike. \\ \hline
  $V_{rev}^E$ & 0 & mV & Reverse potential for excitatory input. \\ \hline
  $V_{rev}^I$ & -70 & mV & Reverse potential for inhibitory input. \\ \hline
  $V_{thr}$ & -50 & mV & Spike threshold. \\ \hline
  $\tau_m$ & 20 & ms & Membrane time constant. \\ \hline
  $\tau_{refrac}$ & 0.1 & ms & Duration of the refractory period. \\ \hline
  $\tau_{syn}^E$ & 5 & ms & Decay time of the excitatory synaptic conductance. \\ \hline
  $\tau_{syn}^I$ & 5 & ms & Decay time of the inhibitory synaptic conductance. \\  \hline
\end{tabular}
\caption{The names, default values and description of the neuron parameters in
PyNN, NEST and BrainScaleS.}
\label{tab:neuron_parameters}
\end{table}

With the exception of $I_{offset}$, which is the constant input current,
the parameters are kept constant in Volr, and used in all spiking backends
to avoid spurious influences.


\section{Parameter translation} \label{sec:translation}

As described in Section \ref{sec:coding} it is necessary to discover the exact
linear relationship between \gls{ANN} activations and \gls{SNN} activations.
Figure \ref{fig:spike_rates} plots the spike count and spike rate against the
constant input current, using the neuron parameters from Table \ref{tab:neuron_parameters}.

\begin{figure}
  \centering
  \includegraphics[width=\textwidth]{images/spike_rate.png}
  \caption{Spike count and spike rates in a single neuron simulated over 50 ms.
  A linear regression ($r^2$ = 0.9977) shows the best-fit linear model.}
  \label{fig:spike_rates}
\end{figure}

The relationship shows that there is an approximated linear correlation, when the
input current is kept below 12 and above 1.
Outside this range the relationship becomes unstable: towards 0 it flatlines and
produces no spikes, and towards and beyond 12 it begins to resemble a non-differentiable 
step function.

To illustrate this point in a deeper network, Figure \ref{fig:spike_rates2} shows
a deeper network where three populations of a single neurons are chained.
The weights have been adjusted using an approximation of the weight normalisation scheme
from \textcite{Rueckauer2017}, shown in Equation \ref{eq:weight_norm}.
The approximation assumes that all biases are 0 and weights are 1, such that
the `pure' activation is the input current, shown in the x axis.

\begin{figure}
  \centering
  \includegraphics[width=\textwidth]{images/spike_rate2.png}
  \caption{Spike count for second and third population in a chained network of
  single-neuron populations, adjusted for previous neuron activation.}
  \label{fig:spike_rates2}
\end{figure}

The code for the above modelling and representations are available in Appendix
\ref{app:verification}.


\section{Problem sets}
Three problem sets will be tested: the NAND ($\neg(A \land B)$) and XOR
($\oplus$) logical gates, as well as the 
Modified National Institute of Standards and Technology
(MNIST) database\index{MNIST}.
The NAND and XOR problems are trivial for \glspl{ANN} to learn, and are used as
a means to test and compare the rudimentary learning capacities of the NEST and
BrainScaleS backends.

The NAND and XOR experiments will be based on the same network topology
(\texttt{\textbf{dense} 2 4 $\obar$ \textbf{dense} 4 2}). 
All backends will execute the experiment with randomly initialised weights. However, the spiking backends will be evaluated
a second time with imported weights and biases from the optimised Futhark
networks.
This is interesting because Futhark is expected to outperform the \glspl{SNN}, and since the
network topology is shared, network parameters can be inserted 1:1.
In theory this should improve the initial training of the spiking models and
lead to an increased accuracy.

The weights and biases from the optimised Futhark model will only be imported into NEST,
which then trains the weights to fit the spiking neuron model.

The MNIST dataset is a widely used for training neural networks to classify
images of digits between 0 and 9. 
It is also commonly used for implementation benchmarks \cite{Schmidhuber2014,
Schmitt2017}, with the best networks scoring an error rate of 0.21\%
\cite{LeCun2019}.
MNIST consists of a collection of 60,000 training images and 10,000 testing images of handwritten digits \cite{LeCun1998}.

To predict the MNIST digits two networks will be constructed.
MNIST images contain 784 pixes (28x28), but to avoid too complex simulations
it is necessary to limit the network size.
The images have been cropped and scaled to 10x10 pixels, such that the initial
network layer can be scaled to 100 neurons.
The topology for the sequential model is \texttt{\textbf{dense} 100 100 $\obar$ \textbf{dense} 100 10}.

To test the parallel structures of the \gls{DSL}, a second and parallel network
will be constructed.
The network will resemble the sequential model, but consist of two separate
parallel subnetworks (\texttt{\textbf{dense} 20 10}), that is merged to produce
an output of 20 neurons.
The full model is as follows:
\texttt{\textbf{dense} 100 20 $\obar$ (\textbf{dense} 20 10 $\ominus$\
\textbf{dense} 20 10) $\obar$ \textbf{dense} 20 10}.
The idea of the model is that the two parallel subsystems can learn semantically
different tasks, and the final layer will be able to `choose' which subnetwork to
use, based on its weights.

\section{Experiment method}
All the above mentioned experiments are classification tasks, and the labels
are encoded as one-hot vectors.
To compare the network output with the labels, the argmax value of the network
output is taken and converted to a one-hot vector of the same shape as the label
data.

To avoid one-off effects such as local minima or (un)fortunate weight
initialisation, all experiments have been repeated 10 times.
The results reported below are accumulations of the prediction accuracies
and errors from the runs.

Weights have been initialised in the models using a normal distribution with a
mean of 1 and a standard deviation of 1.

The experiments use a 80/20 training/testing split with a fixed learning rate of
0.1, and the batch size has been set to 64.

To make the experiments as reproducible as possible, they have all been
initialised with constant random seeds.
Since all randomness in Futhark is based on this seed, all results are constant and
standard deviations are effectively 0.
This is not the case in PyNN, where the randomness is highly backend-specific.
A configuration for setting the initial seed exists
(\texttt{rng\_seed\_seeds})---and have been set for all experiments---but
PyNN does not fully support the randomness configurations in NEST
\cite{Gewaltig2007}.

\end{document}
