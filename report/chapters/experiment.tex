\documentclass[report.tex]{subfiles}
\begin{document}
This chapter describes the experimental setup of the thesis and aims to clarify the methods with which the hypothesis is tested.

This thesis approaches the simulation of biological learning systems in two ways: as \gls{ANN}s and \gls{SNN}s (see section \ref{sec:hypothesis-criteria}).
\gls{SNN}s have been studied extensively in smaller applications such as single-neuron studies, but less so in larger setups such as learning and classification tasks \autocite{dayan2001, Indiveri2015}.
Conversely \gls{ANN}s have been extensively studied in more complex tasks where architectures such as \gls{DNN} have excelled \autocite{schmidhuber2014, Nilsson2009, russel2007}.

While the hypothesis is centred around \gls{SNN}s, the \gls{SNN} necessary to cope with the complexity of the \gls{REF} model, will be closer to the previously mentioned \gls{DNN} in complexity.
For that reason a comparison of to \gls{ANN}s is interesting because it allows the thesis to tap into the vast amount of research on \gls{ANN} while providing a context for traditional machine learning metrics like learning rate and accuracy, regardless of the emerging properties of the REF model \autocite{schmidhuber2014, russel2007}.
\footnote{This relates to the second criteria of the hypothesis (see section \ref{sec:hypothesis-criteria})}.

\gls{ANN}s are, however, architecturally similar to \gls{SNN}s and both are inspired by biological domain \autocite{Nilsson2009, dayan2001}.
To pave the grounds for the experimental setup, this chapter will describe the model that  both the artificial and spiking \gls{NN}s.

 and describes how it projects \gls{NN} models to both ANN and SNN backends.

The section is structured as follows.
First the nature and structure of the experiment is described and motivated.
Second, a practical approach to the modelling of neuronal networks is discussed.
Lastly the three backends on which the model will be evaluated are presented:
  a "regular" machine learning neural network, a neuronal network simulator and a neuromorphic hardware platform.

\section{Krechevsky maze experiment}
To test the hypothesis described in section \ref{sec:hypothesis}, a maze experiment will be constructed.
Mazes have been employed in numerous animal experiment because of their small scale and limited scope \autocite{Krechevsky1932,Tolman1927}.
To encompass the diverse nature of the experiments a number of temporal and spatial arrangements have been devised \autocite{Tolman1927}.
A complicated maze is however outside of the scope of this thesis, since the maze features will have to be modelled in environments with limited capacity for complexity; particularly some neuromorphic hardware limits the degree of experimental complexity (see section \ref{sec:brainscales-limits}).
Further, it is desirable for the human experimenter to introspect the maze and the choices of the agents within the maze.
This would be practically impossible with a large and complex maze setup.
For these reasons the experiment is built around a simple Krechevsky T-maze \ref{Krechevsky1932}.
The maze is is sufficiently simply to be modelled in the experiment environments, while still retaining its relevance for the hypothesis.

A Krechevsky maze is dubbed after its inventor who experimented with rats in the early 20th century.
The maze is built as a T-maze with a series of consecutive left-right corridors.
Taking a wrong choice leads to a dead end, while the right choice leads to the next junction and eventually the exit (with a reward for the rats).
The depth of the maze denotes the amount of choices to make before finding the exit.
A significant property of a Krechevsky maze is the categorial features associated with each junction (e. g. light/darkness, colour or shape).
One task could be for the rat to learn that the colour \texttt{red} consistently leads to a dead end, or that \texttt{darkness} consistently leads to the exit.
More sophisticated, a maze could combine features (\texttt{red} and \texttt{darkness}) or let them alternate (first \texttt{red} then \texttt{darkness}) to lead the rat to its reward.

In the digital realm these features can be encoded using a vector of boolean flags.
For the current experiment this is exploited to construct three digital Krechevsky mazes with one or more binary features.
The three mazes will be build with different levels of complexity: a maze with a depth of one with a single feature, to verify that all models are capable of learning above chance, and two two-feature mazes with a depth of 10.
The second maze (the first two-feature maze) will display each feature one at the time, never showing both features at once.
To learn this pattern requires a temporal memory of the actor.
The third maze displays the features so as to alternate between showing the correct and wrong path.
Specifically the first feature will show the correct path in the first junction, but the incorrect path at the second junction, and vice-versa for the second feature.

The logic of the three mazes will be adapted to suit the different environments described below.

\section{Modelling learning systems} \label{sec:learning}
This section focuses on the practical challenges in constructing ANN and SNN models.
Specifically the models are required to fulfil a number of criteria.
The SNN model should relate to the structure of networks found in the mammalian brain.
The ANN model should closely resemble the SNN model in structure, to allow for a comparison between the two models.
And naturally both models should display a capacity for learning.

... To be continued when models and theory have progressed ...
%, a model will be built in the learning model DSL Volr \autocite{Pedersen2018:volr-report}.

\subsection{Volr}
Volr is a declarative language designed to model \gls{NN}.
Volr offers a trade-off between complete, but verbose, descriptions of small networks and more general designs of large and complex networks.
The language solves this by separating the topology of the networks from the detailed physiological properties of each neuron or neuron population.
A network is constructed in practice by describing a number of neurons (a \texttt{population}) and the connections of these neurons to other \texttt{populations}.
In Volr the \texttt{populations} can be understood as nodes in a graph, and the \texttt{population} connections as edges.

%Using connection-set algebra, Volr is capable of concisely describing the network topology \autocite{Djurfeldt2012}

In Volr the same model can be projected to networks in different environments; currently as either ANNs, simulated SNNs or hardware-backed SNNs.
%The principles for learning vary (see section \ref{sec:futhark} and \ref{sec:neuromorphic}) but the topology in all cases remain the same.
The remainder of this section describes the principles for the translation between the model and the backends.
Following this, learning in each backend will be discussed.

\subsubsection{Emulating neural networks in Volr}


In other words the resulting emulations are derived from the same description, making it simple to compare them, depending on the principles for learning.
The remainder of the section will
Some principles
Volr further supports detailing the type (inhibitory vs. excitatory), cardinality and weights for connections between \texttt{populations}, but is still missing a number of configuration options.
\footnote{TODO}
If necessary additional information can beeach node or edge can
%experiment the machine learning backend (described in section \ref{futhark}) and neuromorphic backend is

%\subsection{Machine learning backend} \label{sec:Futhark}
%Evaluation of the machine learning

\subsection{Neural network simulation backend}

\subsection{Neuromorphic backend} \label{sec:neuromorphic}
\end{document}
