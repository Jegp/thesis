\documentclass[report.tex]{subfiles}
\begin{document}
This chapter describes the experimental setup that aims to validate the
implementation and test the hypotheses introduced in section \ref{sec:hypotheses}.
Firstly, the assumptions and parameters that are the basis of the simulations are described. The second part elaborates on the datasets and methods that are used to test the hypotheses.

\section{Neuron parameters}
Since the neural network topologies between the backends are shown to be similar,
the parameters related to this topology are shared as well.
By applying the theory from \textcite{Rueckauer2017} (see section
\ref{sec:coding}), this section will explain how input data, network weights,
and network biases can be transferred from \glspl{ANN} to \glspl{SNN}.
It will then proceed to describe the setup with which 
%TODO KK finish above sentence

As explained in Section \ref{sec:coding}, normalised input in \glspl{ANN} can
be inserted into \glspl{SNN} with the help of a linear transformation.
This transformation has been examined empirically, by constructing a simple 
one-neuron network, and injecting a constant current over time.
By running a number of experiments, it is possible to measure the integrated
current in the neuron and the amount of spikes it produces over time.

\begin{figure}
  \centering
  \includegraphics[width=0.8\textwidth]{images/membrane.png}
  \caption{Membrane potential for a LIF neuron given a constant input current of
  2 nA, simulated over 100 ms.}
  \label{fig:membrane}
\end{figure}

Figure \ref{fig:membrane} shows one such experiment, in which the membrane
potential of a single neuron is plotted over time.
The neuron is given a constant input current of 1.3 nA, and 
produces 5 spikes during the 100 ms simulation time. 
As expected in section \ref{sec:coding}, the interspike interval is constant.

A neuron only spikes when its membrane potential exceeds its excitation threshold
$V_{thr}$, but depends on a number of parameters to describe the neuron
conductivity, current decay time etc.
In ll three, PyNN, NEST and BrainScaleS, these parameters can be programatically
defined. Table \ref{tab:neuron_parameters} shows the default parameters for
NEST (also shown in Listing \ref{lst:lif-cond-exp}).
The $\tau_{syn}$ parameters denote the decay time for the input spike currents. 
Similarly, the membrane time constant, $\tau_m$, expresses the time it takes for the
neuron membrane to decay to its resting state ($V_{rest}$) if no other
input arrives. 
For the LIF\index{neuron model!leaky-integrate-and-fire} model used in PyNN,
all $\tau$ parameters decay exponentially \cite{Davison2009}.
The $V_{rev}$ parameters explain the potential to integrate into the neuron,
when either excitatory or inhibitory input arrives. 

\renewcommand{\arraystretch}{1.3} 
\begin{table}
\begin{tabular}[center]{l r l l}
  \hline
  $C_m$ & 1 & nF & Capacity of the membrane. \\ \hline
  $I_{offset}$ & 0 & nA & Offset current. \\ \hline
  $V_{rest}$ & -65 & mV & Resting membrane potential. \\ \hline
  $V_{reset}$ & -65 & mV & Reset potential after a spike. \\ \hline
  $V_{rev}^E$ & 0 & mV & Reverse potential for excitatory input. \\ \hline
  $V_{rev}^I$ & -70 & mV & Reverse potential for inhibitory input. \\ \hline
  $V_{thr}$ & -50 & mV & Spike threshold. \\ \hline
  $\tau_m$ & 20 & ms & Membrane time constant. \\ \hline
  $\tau_{refrac}$ & 0.1 & ms & Duration of the refractory period. \\ \hline
  $\tau_{syn}^E$ & 5 & ms & Decay time of the excitatory synaptic conductance. \\ \hline
  $\tau_{syn}^I$ & 5 & ms & Decay time of the inhibitory synaptic conductance. \\  \hline
\end{tabular}
\caption{The names, default values and description of the neuron parameters in
PyNN, NEST and BrainScaleS.}
\label{tab:neuron_parameters}
\end{table}

With the exception of $I_{offset}$, which is the constant input current,
the parameters are kept constant in Volr, and used in all spiking backends
to avoid spurious influences.


\section{Parameter translation}

As described in Section \ref{sec:coding} it is necessary to discover the exact
linear relationship between \gls{ANN} activations and \gls{SNN} activations.
Figure \ref{fig:spike_rates} plots the spike count and spike rate against the
constant input current, using the neuron parameters from Table \ref{tab:neuron_parameters}.

\begin{figure}
  \centering
  \includegraphics[width=\textwidth]{images/spike_rate.png}
  \caption{Spike count and spike rates in a single neuron simulated over 50 ms.
  A linear regression ($r^2$ = 0.9977) shows the best-fit linear model.}
  \label{fig:spike_rates}
\end{figure}

The relationship shows that there is an approximated linear correlation, when the
input current is kept below 12 and above 1.
Outside this range the relationship becomes unstable: towards 0 it flatlines and
produces no spikes, and towards and beyond 12 it begins to resemble a non-differentiable 
step function.

To illustrate this point in a deeper network, Figure \ref{fig:spike_rates2} shows
a deeper network where three populations of a single neurons are chained.
The weights have been adjusted using an approximation of the weight normalisation scheme
from \textcite{Rueckauer2017}, shown in Equation \ref{eq:weight_norm}.
The approximation assumes that all biases are 0 and weights are 1, such that
the `pure' activation is the input current, shown in the x axis.

\begin{figure}
  \centering
  \includegraphics[width=\textwidth]{images/spike_rate2.png}
  \caption{Spike count for second and third population in a chained network of
  single-neuron populations, adjusted for previous neuron activation.}
  \label{fig:spike_rates2}
\end{figure}

The code for the above modelling and representations are available in Appendix
\ref{app:verification}.


\section{Problem sets}
Three problem sets will be tested: the NAND ($\neg(A \land B)$) and XOR
($\oplus$) logical gates, as well as the 
Modified National Institute of Standards and Technology
(MNIST) database\index{MNIST}.
The NAND and XOR problems are trivial for \glspl{ANN} to learn, and are used as
a means to test and compare the rudimentary learning capacities of the NEST and
BrainScaleS backends.

The NAND and XOR experiments will be based on the same network topology
(\texttt{\textbf{dense} 2 4 $\obar$ \textbf{dense} 4 2}). 
All backends will execute the experiment with randomly initialised weights. However, the spiking backends will be evaluated
a second time with imported weights and biases from previously optimised networks, because the experiments are structurally similar, and Futhark is
expected to outperform the \glspl{SNN}.
The weights and biases from the optimised Futhark model will be imported into NEST,
and, after discretising them, the optimised weights and biases from NEST will be
inserted into the BrainScaleS model.
Because of the imprecision in the coding translation scheme between \glspl{ANN} and
\glspl{SNN}, this is expected to improve the performance of the \glspl{SNN} significantly.

%TODO KK check the following sentence. Looks weird, but maybe it'S fine?
The MNIST network will 
\texttt{\textbf{dense} (\textbf{dense} 2 2 $\ominus$ \textbf{dense} 2 2) $\obar$ \textbf{dense} 4 2}.
The two networks exist to compare the performance of the parallel structure with 
a traditional sequential one.

The MNIST dataset is a widely used for training neural networks to classify digits (0 - 9),and is commonly used for implementation benchmarks \cite{Schmidhuber2014, Schmitt2017}. It is a famous collection of 60,000 training images and 10,000 testing images of handwritten digits \cite{LeCun1998}.

To solve this problem set, a sequential and a parallel network will be constructed:
\texttt{\textbf{dense} 100 20 $\obar$ \textbf{dense} 20 10} and
\texttt{(\textbf{dense} 100 10 $\ominus$ \textbf{dense} 100 10) $\obar$ \textbf{dense 20 10}}.

\begin{comment}

\subsection{Neuromorphic backend} \label{sec:neuromorphic}
% \subsubsection{Experiment stimuli}
% The stimuli describes the ``input'' of the model.
% Such input is defined either as an array of elements directly in the DSL
% or as a reference to a file.

% \subsubsection{Experiment populations}
% The populations describe the topology of the neural network itself.
% As with the stimuli, the populations are built around a block structure that
% contains a number of sub-expressions.

% The \texttt{connection} defines the source stimulus for the population,
% i.e. the population \textit{from} which action-potentials will be forwarded.
% A population can receive stimulus from more than one source.
% The connections are modelled as per the \gls{CSA} described in section
% \ref{sec:volr-csa}.

% % TODO: Describe and invent archetypes... or not?

% \subsubsection{Experiment responses}
% The responses are the ``output'' of the model to be recorded, and can be
% considered as the outcome of the network for training purposes.
% The response block only contains an optional specification of a location for
% the experiment output data.

\end{comment}
\end{document}
