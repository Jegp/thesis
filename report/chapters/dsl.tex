\documentclass[report.tex]{subfiles}
\begin{document}

This first chapter of the thesis motivates and defines
a \gls{DSL} for the modelling of general \gls{NN}s.
The first part of the chapter presents the theory behind second and third
generation \gls{NN}s, so as to posit requirements for the language. 
Finally this chapter will present the \gls{DSL} \index{Volr},
as a means to translate cognitive concepts into computational \gls{NN}
models.

\section{Neural networks}
\Gls{NN}s is a broad term that originates in the neuronal models from
biological brain \cite{Dayan2001}.
The general architecture of neural systems can be explained as circuits
of neurons \index{neuron} connected through weighted edges.
\cite{Russel2007, Dayan2001}.
In this abstract sense a neuron is defined as a computational unit that
takes a number of signals (inputs) and processes them through some
function $f$ that outputs a single value \cite{Eliasmith2004}.
From that perspective neural networks simply \textit{computes} an 
output based on some input why neural networks can be understood as
complex non-linear computations \cite{Eliasmith2004, Dayan2001}.

In a more concrete sense neural networks computes over either
continuous (e. g. voltage or numbers) or discrete signal, and they
can be modelled with or without a temporal dimension
\cite{Eliasmith2004, Nilsson2009}.

Discrete models without a temporal dimension were the foundation for
the first perceptron models, as seen in equation \ref{eq:perceptron}.

\begin{figure}
  
\end{figure}

\gls{ANN} this design has been 
Several variations of neural models exist within neuroscience, but they

The artificial flavour is 
circuits of computational
units connected through weighted edges \cite{Dayan2001, Eliasmith2004}.

% TODO: Distributions of properties instead of actual properties

% Both types of network are architecturally similar, and both are conceived from
% the same physiological principles \autocite{dayan2001, russel2007, Nilsson2009, schmidhuber2014}.
% The implementations, however, vary greatly.

% To ensure internal and external validity in and between the two network types,
% it is desirable that the models are as closely related from a theoretical and
% practical perspective as possible.
% Additionally, to test the hypothesis, it is required that both the artificial
% and spiking models can be simulated on regular machine architecture, while
% the spiking model requires a neuromorphic hardware platform.

% An optimal approach would be to find a tool that leverages the similarities
% of the network types, while integrating with the diverse simulated or emulated
% targets.
% That is, an abstract model of neural networks that can translate into
% heterogeneous back-ends, while retaining a high degree of inter-model validity.

% A number of general frameworks for artificial neural networks
% exist\footnote{
%   Among others, see \autocite{ONNX2018}, \autocite{PyTorch2018}, \autocite{TensorFlow2018},
%   \autocite{Keras2018} and DyNet \autocite{Neubig2017}.
% }, but none of them extend to the spiking domain.
% Conversely a number of choices exist for neuromorphic modelling\footnote{
%   %TODO: Find sources on internal IBM/Intel stuff
% }, but they exclusively evaluate to neuromorphic or spiking neural network
% backends \autocite{Jordan2018}.

\subsection{Learning in neural networks}

\subsubsection{Backpropagation}

\section{DSL requirements}

\section{Similar work}

This sections presents the \gls{DSL} Volr. \index{Volr}
The main purpose of Volr is to define clear and reproducible
experiments whose semantics are retained regardless of
the runtime environment.
The specification in its current form is relatively simple, but sufficiently 
complicated for the purpose of this thesis.
It focuses solely on the topology of networks, thus
separating the network description from any generation-specific properties
of neurons or neuron populations.

The first requirement is achieved through an unambiguous syntax inspired
by the lamdba calculus \cite{Pierce2002}.
Figure \ref{fig:volr-expr} shows the BNF notation for expressions, values and types
in Volr. 
Figure \ref{fig:volr-rules} lists evaluation rules for the correct
interpretation of the expressions.

% Expression figure
\begin{figure}
  \begin{tabular}[t]{l l}
    expressions & \texttt{$e$ ::= $n$} \\
    & \begin{minipage}{0.6\textwidth}
      \begin{Verbatim}[mathescape,commandchars=\\\{\}]
    | \textbf{dense} $n\ m$
    | \textbf{let} $x = e$ \textbf{in} $e'$
    | $e\ \obar\ e'$
    | $e\ \ominus\ e'$
    | $\neg e$
      \end{Verbatim} 
      \end{minipage} \\

    & \\ % Empty space 

    values
    & \texttt{$v$ ::= $\textbf{net}\ n\ m$} \\
    
    & \\ % Empty space
    types
    & \texttt{$\tau$ ::= \textbf{int} | \textbf{net} $n\ m$} \\
  \end{tabular}

  \caption{Expressions, values and types of the Volr language.}
  \label{fig:volr-expr}
\end{figure}
\begin{figure}
\begin{prooftree}
  \AxiomC{}
  \UnaryInfC{$\Gamma \vdash n : \mathbf{int}$}
\end{prooftree}
\begin{prooftree}
  \AxiomC{}
  \UnaryInfC{$\Gamma \vdash \mathbf{stim}\ n : \mathbf{layer}\ n$}
\end{prooftree}
\begin{prooftree}
  \AxiomC{}
  \UnaryInfC{$\Gamma \vdash \mathbf{pop}\ n : \mathbf{layer}\ n$}
\end{prooftree}
\begin{prooftree}
  \AxiomC{$\Gamma (x) = \tau$}
  \UnaryInfC{$\Gamma \vdash x : \tau$}
\end{prooftree}
\begin{prooftree}
  \AxiomC{$\Gamma \vdash e_1 : \mathbf{layer}\ n$}
  \AxiomC{$\Gamma \vdash e_2 : \mathbf{layer}\ m$}
  \AxiomC{$\Gamma \vdash w : \mathbf{float}$}
  \TrinaryInfC{$\Gamma \vdash \otimes\ e_1\ e_2\ w : \mathbf{con}\ n\ m$}
\end{prooftree}
\begin{prooftree}
  \AxiomC{$\Gamma \vdash e_1 : \mathbf{layer}\ n$}
  \AxiomC{$\Gamma \vdash e_2 : \mathbf{layer}\ m$}
  \AxiomC{$\Gamma \vdash w : \mathbf{float}$}
  \TrinaryInfC{$\Gamma \vdash \ominus\ e_1\ e_2\ w : \mathbf{con}\ n\ m$}
\end{prooftree}
\begin{prooftree}
  \AxiomC{$\Gamma \vdash e : \tau$}
  \AxiomC{$\Gamma [v : \tau] \vdash e' : \tau$}
  \BinaryInfC{$\Gamma \vdash \mathbf{let}\ x = e\ in\ e' : \tau$}
\end{prooftree}

  \caption{Evaluation rules in Volr.}
  \label{fig:volr-rules}
\end{figure}



The constant expression $n$ is simply an integer that evaluates to the type 
\texttt{\textbf{int}} ($e1$). 
Similarly to the lambda calculus, the \texttt{\textbf{let}} binding binds
the string constant $x$ to the expression $e$ in an encapsulated
environment $e'$ \cite{Pierce2002}.
That constant can later be referenced in the $e'$ expression 
through the string $x$ as shown in $e2$.

The \texttt{\textbf{dense}} expression describes a network of two
populations, and is the most basic concept in the \gls{DSL}.
Notice the distinction between a neural network \textit{layer}
such as \texttt{\textbf{dense}} and a population. 
In a \texttt{\textbf{dense}} network layer, every neuron from the
first population is connected to every neuron in the second population
(\textit{densely} or all-to-all).
The two parameters $n$ and $m$ defines the number of neurons in the first and 
second layer respectively, and evaluates to the \texttt{\textbf{net}}
fundamental network type and value, as shown in $e3$. 
Considering how each neuron is a type of classifier, these numbers
illustrate the \textit{dimensionality} of the network, such that the number of
dimensions in the input is truncated (or expanded) to classify the
dimensionality of the output layer.

The $\obar$ (sequential) operator binds two networks sequentially,
such that the output layer of the first network becomes the input layer 
of the second network.
The two networks $e$ and $e'$ will share one of the neuron populations, why
the output size of the first layer is required to be equal to the input
size of the second layer ($e4$).

The $\ominus$ (parallel) operator parallelises two networks by duplicating
the input from the previous layer and merging the outputs into a single
layer ($e5$).
The input feeds into both $e$ and $e'$, such that the input dimension of
the network must be shared by the two layers ($l$). 
The output from the network is stacked such that each neuron from each
population corresponds to one output neuron ($e_{out} + e'_{out}$).
This is done to preserve the meaning of each parallel population.

Taken together these constructs can express simple neural networks and
the properties of their connections. 
Figure \ref{fig:volr-examples} shows a number of example networks
that visualises four examples of networks. 

\begin{figure}
  \ContinuedFloat*
  \begin{tabular}[t]{c c}
    \begin{minipage}{0.5\textwidth}
      \begin{Verbatim}[mathescape,commandchars=\\\{\}]
\textbf{let} s = stim 2 \textbf{in}
  \textbf{let} p = pop 2 1 \textbf{in}
    s $\otimes$ p
      \end{Verbatim}
    \end{minipage} & \begin{minipage}{0.5\textwidth}
      \includegraphics[width=\textwidth]{chapters/volr/example1.pdf}
    \end{minipage}

  \end{tabular}
  \caption{A textual and visual example of a network with a stimulus,
    a population and an implicit output.}
  \label{fig:volr-example1}
\end{figure}

\begin{figure}
  \ContinuedFloat*
  \begin{tabular}[t]{c c}
    \begin{minipage}{0.5\textwidth}
      \begin{Verbatim}[mathescape,commandchars=\\\{\}]
\textbf{let} s = stim 2 \textbf{in}
  \textbf{let} p = pop 2 1 \textbf{in}
    s $\otimes$ p
      \end{Verbatim}
    \end{minipage} & \begin{minipage}{0.5\textwidth}
      Hello
    \end{minipage}
  \end{tabular}
\label{fig:volr-examples}
\end{figure}

\FloatBarrier

\end{document}
