\documentclass[report.tex]{subfiles}
\begin{document}

This chapter presents the \gls{DSL} Volr \index{Volr}.

Before presenting and specifying the language itself, 
it is beneficial to examine existing work within the
programming and simulation of second and third
generation \gls{NN}s.
Following the existing work, a detailed list of requirements
are provided to accurately scope the \gls{DSL}.
Finally, the implementation of the \gls{DSL} is presented and
accompanied by code and test examples.

\section{Similar work}
A vast amount of work has been put into the development of software for simulating
neural networks.
This section covers recent work within the simulation of second and third
generation \glspl{NN}, and extracts relevant findings for use in the
requirements section \ref{sec:requirements}.

It is likely that the following list is incomplete due to the fragmented and 
fast-paced nature of the field.
To encompass as many relevant findings as humanly possible, the subsequent
references are based on a number of source, primarily based on books
(\cite{Bishop2006, Russel2007, Eliasmith2015, Lin2018, Nilsson2009, Pearl1988, Rojas1996, Rumelhart1988}), review papers 
(\cite{Schmidhuber2014, Blundell2018, Markram2013, Walter2015, Hunsberger2015}),
and, of course, articles, which will be referenced below.
%TODO:  https://github.com/Shikhargupta/Spiking-Neural-Network

\subsection{Second generation}
The perhaps most notable product for this type of networks is the Tensorflow \index{Tensorflow}
framework \cite{Abadi2016}.
Tensorflow is an \gls{API} for the description and execution of directed graph 
structures,
that connects varying activation functions and learning mechanisms through the common abstraction
of tensors \cite{Abadi2015}.
Tensorflow is the result of a large collaboration of multiple companies and organisations, who have
developed a comprehensive library of both code as well as infrastructure, and they
provide extensive support for hardware acceleration \cite{Abadi2015}.

The primary advantage of Tensorflow \index{Tensorflow} arrives from 
its foundation in tensors as a general abstraction, that
can be applied to a wide array of problems \cite{Abadi2016}.
Other frameworks have adapted a similar approach, such as PyTorch \cite{PyTorch2018}, 
scikit-learn \cite{Sklearn2018}, Microsoft Cognitive Toolkit (CNTK) \cite{CNTK2018},
Caffe \cite{Caffe2018} and Theano \cite{Theano2018}.

Lasagne and Keras are built on top of Theano and Tensorflow respectively,
and works similarly well higher-level abstractions \cite{Lasagne2018, Keras2018}.
They both provide imperative \gls{API}s for constructing models in steps, while
including useful utilities for preprocessing of data.

Where scikit-learn, CNTK, Tensorflow and Theano targets \glspl{NN} in general,
PyTorch and Caffe are frameworks that specifically target \gls{DL} 
\cite{PyTorch2018, Caffe2018}.
All of them, however, rely on second generation \gls{NN} architecture in a general sense.

To provide a comparison between them, a number of examples are provided below.
The Listing \ref{code:caffe} shows a network with a single fully connected
layer in Caffe, built to recognise handwritten digits from the 
popular MNIST dataset \cite{LeCun1998}.
Caffe is verbose compared to the full network definitions in PyTorch 
(Listing \ref{code:pytorch}), Keras (Listing \ref{code:keras}) and
Lasagne (Listing \ref{code:lasagne}), but provides additional 
configuration options for setting of weights and biases in 
individual layers.

\lstset{caption={A network layer for the MNIST task in Caffe.}}
\begin{lstlisting}
layer {
  name: "ip1"
  type: "InnerProduct"
  param { lr_mult: 1 }
  param { lr_mult: 2 }
  inner_product_param {
    num_output: 500
    weight_filler { type: "xavier" }
    bias_filler { type: "constant" }
  }
  bottom: "pool2"
  top: "ip1"
}
\end{lstlisting} \label{code:caffe}

\lstset{language=Python,caption={MNIST network in PyTorch.}}
\begin{lstlisting}{language=Python}
self.fc1 = nn.Linear(28*28, 128)   # Topology
self.fc2 = nn.Linear(128, 10)
x = x.view(-1, 28, 28)             # Activations
x = F.relu(self.fc1(x))
X = self.fc2(x)
\end{lstlisting} \label{code:pytorch}

\lstset{language=Python,caption={MNIST network in Tensorflow using the Keras API.}}
\begin{lstlisting}
keras.Sequential([
  keras.layers.Flatten(input_shape=(28, 28)),
  keras.layers.Dense(128, activation=tf.nn.relu),
  keras.layers.Dense(10, activation=tf.nn.softmax)
])
\end{lstlisting} \label{code:keras}

\lstset{language=Python, caption={MNIST network in Theano using the Lasagne API.}}
\begin{lstlisting}
l_in = lasagne.layers.InputLayer(shape=(None, 1, 28, 28),
           input_var=input_var)
l_hid = lasagne.layers.DenseLayer(l_in, num_units=128,
           nonlinearity=lasagne.nonlinearities.rectify)
l_out = lasagne.layers.DenseLayer(
           l_hid, num_units=10,
           nonlinearity=lasagne.nonlinearities.softmax)
\end{lstlisting} \label{code:lasagne}

In terms of learning the frameworks are diverse, although gradient descent 
and auto-differentiation (where transformations are automatically registered and 
later derived using the chain rule) are among the most common features 
(seen in Tensorflow, PyTorch, CNTK, Theano and Caffe). 

The Open Neural Network Exchange Format (ONNX) is an open data format for the representation
of \gls{DL} learning models \cite{ONNX2018}. 
ONNX is interesting in this context because it, like all the frameworks above, describes 
networks as directed graphs, defined by nodes of a certain dimension (shape) connected through
edges with certain activations (operations).

\subsubsection{Neural networks in Futhark}
\index{Futhark}
The data-parallel language Futhark is designed to produce efficient parallel
code, suitable for parallel application \cite{Henriksen2017}.
Second generation \glspl{NN} are such an application, because they can be reduced
to a large number of parallel matrix multiplications.

Futhark supports hardware acceleration through the \gls{OpenCL} framework,
and can integrate with other languages, such as \gls{Python} through the
PyOpenCL interface \cite{PyOpenCL}.

\citeauthor{Minh2018} developed a library for machine learning, which 
allows the construction of densely connected layers and gradient descent
training \cite{Minh2018}.

\subsection{Third generation}
The landscape for third generation software is less homogeneous, first and
foremost because the field is still young \cite{Maass1997}.
Secondly there are two different approaches to the evaluation of \gls{SNN}s:
through simulation on general purpose hardware or specialised analogue
(neuromorphic\index{neuromorphic}) hardware \cite{Maass1997, Davison2009, Albada2018}.

For each platform---digital or otherwise---a complete programming environment
is developed from scratch, because of the degree of specialisation
\cite{Walter2015, Lin2018}.
This section covers the most important technical details of
the environments.

\subsubsection{\Gls{SNN} simulators} \label{sec:SNN-simulators}
Based on the review of \textcite{Blundell2018} this paper will discuss
the following third generation \gls{SNN} simulators: PyNN \cite{Davison2009},
NEST \cite{Gewaltig2007}, NEURON \cite{Carnevale2007},
Brian \cite{Goodman2013} and Nengo \cite{Eliasmith2015}.

PyNN is a ``simulator-independent language''
\cite{PyNN2018} that compiled to both simulated and
accelerated architectures \cite{Davison2009}.
Technically PyNN is not a simulator but acts as an interface to any third generation
backend \cite{Davison2009}.
PyNN currently interfaces with Brian, NEST, BrainScaleS and SpiNNaker
\index{Brian}\index{NEST}\index{BrainScaleS}\index{SpiNNaker}
and is more than 10 years old, \cite{Davison2009} older
than most neuromorphic chips. 
Their \gls{API}s were designed a priori and lacked a number of crucial
elements,\footnote{In particular in relation to the mapping of neuron population
onto the physical chip, see \ref{sec:Cairo}.} which the hardware designers sought
to resolve by augmenting the interface \cite{Pfeil2013, PyNN2018}.
The result is a fragmented environment that supports basic morphologies, 
but where each experiment requires retrofitting to execute correctly on the 
backends \cite{PyNN2018}.

Nengo is a neural simulation environment for large-scale neural models, with
a focus on graphical modelling \cite{Eliasmith2015}. 
The Nengo project bases itself on the Neural Engineering Framework (NEF)
\index{Neural Engineering Framework} that offers a concise language for
describing third generation simulations \cite{Bekolay2014}, along with
the limited rendering of logical computations (such as logical gates and basic
mathematical functions) into approximated
\gls{NN} structures \cite{Eliasmith2004, Eliasmith2015}.
Nengo supports a wide range a backends---non-spiking networks through Tensorflow,
simulated spiking networks through its custom \gls{OpenCL} engine and 
hardware accelerated networks through the neuromorphic platform SpiNNaker---
but have a limited repertoire of models compared to other simulators
\cite{Nengo2018}.

Listings \ref{code:Nengo} and \ref{code:PyNN} shows how
basic spiking models are defined using Nengo and PyNN.

\begin{minipage}{\linewidth}
\lstset{caption={A simple LIF MNIST population network in Nengo.}, label=code:Nengo}
\begin{lstlisting}
pop_1 = nengo.Ensemble(nengo.LIF(100), 2)
pop_2 = nengo.Ensemble(nengo.LIF(10),  1)
nengo.Connection(pop_1, pop_2)
\end{lstlisting}
\end{minipage}

\begin{minipage}{\linewidth}
\lstset{caption={A simple LIF MNIST network in PyNN.}, label=code:PyNN}
\begin{lstlisting}
pop_1 = nest.Create('iaf_exp_cond', 100)
pop_2 = nest.Create('iaf_exp_cond',  10)
nest.Connect(pop_1, pop_2, 'all_to_all')
\end{lstlisting} 
\end{minipage}
\\[0.3cm]
PyNN and Nengo both attempt to converge platform differences into one single \gls{API}, and
offer high-level description of networks with support for detailed 
configuration.
Nengo also offers an approximated model that can be evaluated in Tensorflow
\cite{Hunsberger2015}, but it does not translate to other simulators like PyNN:
models written for one simulator cannot be transferred to another \cite{Nengo2018}.

The NEST simulator supports both point neurons \index{neuron!point}
and compartmentalised models to support speed as well as sophisticated neuron
geometry \cite{Gewaltig2007}.
NEST claims to focus on ``dynamics, size and structure rather than on the detailed
morphological and biophysical properties of individual neurons''
\cite{Gewaltig2007}.
It has modelled a large number of neuron models and optimisations
\cite{Blundell2018}.

NEURON targets complex and detailed simulations of multi-chamber models, and
attempts to model all aspects of the biophysical properties \cite{Carnevale2007}.
Brian strikes somewhere between NEST and NEURON because it allows users to
inject their own models through custom equations in plain text \cite{Goodman2013}.

\textcite{Rueckauer2017} implemented a ``Spiking neural network conversion
toolbox'' that converts second generation \glspl{NN} into \glspl{SNN}.
As shown in Section \ref{sec:coding} they approach this by estimating the 
firing rate of LIF \index{LIF} neurons through fixed current inputs,
assuming even Poisson\index{Poisson distribution} distributed signals
throughout the network \cite{Rueckauer2017}.

\subsubsection{Neuromorphic hardware} \label{sec:similar-neuromorphic}
Based on the review of \textcite{Walter2015} and the work from
\textcite{Lin2018}, the following section classifies neuromorphic hardware into
two categories: either as digital interpretations of neural components, 
or as analogue emulations of neural tissue.

Digital neuromorphic chips digitises the neural signals and mimics neuron
behaviour either through the regular \gls{vonNeumann}, or
via custom digital components \cite{Walter2015}.
SpiNNaker is an example of the former, where a number of ARM \gls{ARM}
processors that are equipped with controllers for handling timers and
interrupts \cite{Walter2015}.
This permits SpiNNaker to compute arbitrary logic, while retaining
a large degree of parallelism \cite{Albada2018}.
IBM's TrueNorth \index{IBM!TrueNorth} and Intel's Loihi \index{Intel!Loihi}
are examples of neuromorphic hardware with custom digital components 
\cite{Walter2015, Lin2018}.
TrueNorth consist of 4096 independently operating neurosynaptic cores, each implementing 256
digital neurons in silicon \cite{Walter2015, ArtificialBrains2018}.
The Loihi seems similar to the TrueNorth chip, with the difference that
its 128 neuromorphic cores feature programmable
synaptic learning rules \cite{Lin2018}.

Analogue neuromorphic chips construct circuits that equals those of biological
neurons \cite{Walter2015}.
BrainScaleS \cite{Schmitt2017}, Neurogrid \cite{BrainsInSilicon2018}, 
and ROLLS (Reconfigurable On-line Learning Spiking)
\cite{Walter2015} are examples of such chips.

BrainScaleS is built on the High Input Count Analog Neural Network (HICANN)
chip, that contains up to 512 neurons depending on the hardware configuration 
\cite{Pfeil2013}.
Several HICANN chips can be integrated to allow the simulation of larger
networks, where dedicated \gls{FPGA}s set weights for each neuron and
communicate with other \gls{FPGA}s on chip \cite{Walter2015}. 
Neurogrid models around $10^6$ two-compartment neurons, where the dendritic
tree is separated from the neuron `soma' \cite{Walter2015}.
The spikes are transmitted digitally through \gls{RAM} \cite{Walter2015}.
The ROLLS processor consists of 256 analogue silicon neurons with
$\sim1.3 \cdot 10^5$ synapses, but with fixed synaptic weights
\cite{Walter2015}.

\section{DSL requirements} \label{sec:requirements}
This section sketches four functional requirements for a \gls{DSL} that 
fulfils the necessary conditions to allow the testing of the thesis hypothesis.
The requirements steers the specification as defined in section 
\ref{sec:volr} and later the implementation details in section
\ref{sec:implementation}.

\paragraph{1. Semantic consistency}
The overarching goal of the \gls{DSL} is to allow the translation 
of \gls{NN} descriptions into semantically similar models in backend runtime
environments.
In other words a network described in the \gls{DSL} should carry
the same semantic structure when translated to second or third generation
implementations. 

Because of the diverse and incompatible nature of spiking neural network
landscape, this is a non-trivial requirement,
but a necessity one if the models are to be validated across \gls{NN} paradigms.
This requirement is approached empirically, by illustrating examples in
both generations and validate whether they achieved the desired degree
of external validity.

\paragraph{2. Translation to second and third generation}
A second requirement is the translation of the \gls{DSL} into
three runtime environments to permit a sufficient degree of
generalisation.
It is required that the DSL can translate code that evaluate networks
in second generation, simulated third generation, as well as
analog third generation (neuromorphic).

\paragraph{3. Learning}
It is furthermore required that the \gls{DSL} supports a form of
learning, to illustrate the expected theoretical adaptation.
For this purpose, supervised learning through backpropagation\index{backpropagation}
is sufficient for the purpose of this thesis.

\paragraph{4. Well-typed}
As a final functional requirement the \gls{DSL} is designed to ensure
consistency and disallow any networks that are not well-formed at
compile time.
\\[0.4cm]
\noindent
None of the above mentioned environments fulfil all four requirements.
Models built in Nengo and PyNN\index{Nengo}\index{PyNN} can be evaluated
in both second and third generation environments, but Nengo does not 
offer consistent semantics between the backends and PyNN only allows
for a partial translation into the neuromorphic platforms.
However, PyNN does support a consistent \gls{API} to describe models that, at
least topologically, translates to both simulated and accelerated 
backends.

The following two sections will present the design and implementation of
Volr, while drawing on the requirements above to assess whether and
how they are met.

\section{DSL specification} \label{sec:volr}
This sections presents the \gls{DSL} Volr. \index{Volr}
The main purpose of Volr is to define clear and reproducible
experiments whose semantics are retained regardless of
the runtime environment.
The specification in its current form is relatively simple, but sufficiently 
complicated for the purpose of this thesis.
It focuses solely on the topology of networks, thus
separating the network description from any generation-specific properties
of neurons or neuron populations.

The first requirement is achieved through an unambiguous syntax inspired
by the lamdba calculus \cite{Pierce2002}.
Figure \ref{fig:volr-expr} shows the BNF notation for expressions, values and types
in Volr. 
Figure \ref{fig:volr-rules} lists evaluation rules for the correct
interpretation of the expressions.

% Expression figure
\begin{figure}
  \begin{tabular}[t]{l l}
    expressions & \texttt{$e$ ::= $n$} \\
    & \begin{minipage}{0.6\textwidth}
      \begin{Verbatim}[mathescape,commandchars=\\\{\}]
    | \textbf{dense} $n\ m$
    | \textbf{let} $x = e$ \textbf{in} $e'$
    | $e\ \obar\ e'$
    | $e\ \ominus\ e'$
    | $\neg e$
      \end{Verbatim} 
      \end{minipage} \\

    & \\ % Empty space 

    values
    & \texttt{$v$ ::= $\textbf{net}\ n\ m$} \\
    
    & \\ % Empty space
    types
    & \texttt{$\tau$ ::= \textbf{int} | \textbf{net} $n\ m$} \\
  \end{tabular}

  \caption{Expressions, values and types of the Volr language.}
  \label{fig:volr-expr}
\end{figure}
\begin{figure}
\begin{prooftree}
  \AxiomC{}
  \UnaryInfC{$\Gamma \vdash n : \mathbf{int}$}
\end{prooftree}
\begin{prooftree}
  \AxiomC{}
  \UnaryInfC{$\Gamma \vdash \mathbf{stim}\ n : \mathbf{layer}\ n$}
\end{prooftree}
\begin{prooftree}
  \AxiomC{}
  \UnaryInfC{$\Gamma \vdash \mathbf{pop}\ n : \mathbf{layer}\ n$}
\end{prooftree}
\begin{prooftree}
  \AxiomC{$\Gamma (x) = \tau$}
  \UnaryInfC{$\Gamma \vdash x : \tau$}
\end{prooftree}
\begin{prooftree}
  \AxiomC{$\Gamma \vdash e_1 : \mathbf{layer}\ n$}
  \AxiomC{$\Gamma \vdash e_2 : \mathbf{layer}\ m$}
  \AxiomC{$\Gamma \vdash w : \mathbf{float}$}
  \TrinaryInfC{$\Gamma \vdash \otimes\ e_1\ e_2\ w : \mathbf{con}\ n\ m$}
\end{prooftree}
\begin{prooftree}
  \AxiomC{$\Gamma \vdash e_1 : \mathbf{layer}\ n$}
  \AxiomC{$\Gamma \vdash e_2 : \mathbf{layer}\ m$}
  \AxiomC{$\Gamma \vdash w : \mathbf{float}$}
  \TrinaryInfC{$\Gamma \vdash \ominus\ e_1\ e_2\ w : \mathbf{con}\ n\ m$}
\end{prooftree}
\begin{prooftree}
  \AxiomC{$\Gamma \vdash e : \tau$}
  \AxiomC{$\Gamma [v : \tau] \vdash e' : \tau$}
  \BinaryInfC{$\Gamma \vdash \mathbf{let}\ x = e\ in\ e' : \tau$}
\end{prooftree}

  \caption{Evaluation rules in Volr.}
  \label{fig:volr-rules}
\end{figure}



The constant expression $n$ is simply an integer that evaluates to the type 
\texttt{\textbf{int}} ($e1$). 
Similarly to the lambda calculus, the \texttt{\textbf{let}} binding binds
the string constant $x$ to the expression $e$ in an encapsulated
environment $e'$ \cite{Pierce2002}.
That constant can later be referenced in the $e'$ expression 
through the string $x$ as shown in $e2$.

The \texttt{\textbf{dense}} expression describes a network of two
populations, and is the most basic concept in the \gls{DSL}.
Notice the distinction between a neural network \textit{layer}
such as \texttt{\textbf{dense}} and a population. 
In a \texttt{\textbf{dense}} network layer, every neuron from the
first population is connected to every neuron in the second population
(\textit{densely} or all-to-all).
The two parameters $n$ and $m$ defines the number of neurons in the first and 
second layer respectively, and evaluates to the \texttt{\textbf{net}}
fundamental network type and value, as shown in $e3$. 
Considering how each neuron is a type of classifier, these numbers
illustrate the \textit{dimensionality} of the network, such that the number of
dimensions in the input is truncated (or expanded) to classify the
dimensionality of the output layer.

The $\obar$ (sequential) operator binds two networks sequentially,
such that the output layer of the first network becomes the input layer 
of the second network.
The two networks $e$ and $e'$ will share one of the neuron populations, why
the output size of the first layer is required to be equal to the input
size of the second layer ($e4$).

The $\ominus$ (parallel) operator parallelises two networks by duplicating
the input from the previous layer and merging the outputs into a single
layer ($e5$).
The input feeds into both $e$ and $e'$, such that the input dimension of
the network must be shared by the two layers ($l$). 
The output from the network is stacked such that each neuron from each
population corresponds to one output neuron ($e_{out} + e'_{out}$).
This is done to preserve the meaning of each parallel population.

Taken together these constructs can express simple neural networks and
the properties of their connections. 
Figure \ref{fig:volr-examples} shows a number of example networks
that visualises four examples of networks. 

\begin{figure}
  \ContinuedFloat*
  \begin{tabular}[t]{c c}
    \begin{minipage}{0.5\textwidth}
      \begin{Verbatim}[mathescape,commandchars=\\\{\}]
\textbf{let} s = stim 2 \textbf{in}
  \textbf{let} p = pop 2 1 \textbf{in}
    s $\otimes$ p
      \end{Verbatim}
    \end{minipage} & \begin{minipage}{0.5\textwidth}
      \includegraphics[width=\textwidth]{chapters/volr/example1.pdf}
    \end{minipage}

  \end{tabular}
  \caption{A textual and visual example of a network with a stimulus,
    a population and an implicit output.}
  \label{fig:volr-example1}
\end{figure}

\begin{figure}
  \ContinuedFloat*
  \begin{tabular}[t]{c c}
    \begin{minipage}{0.5\textwidth}
      \begin{Verbatim}[mathescape,commandchars=\\\{\}]
\textbf{let} s = stim 2 \textbf{in}
  \textbf{let} p = pop 2 1 \textbf{in}
    s $\otimes$ p
      \end{Verbatim}
    \end{minipage} & \begin{minipage}{0.5\textwidth}
      Hello
    \end{minipage}
  \end{tabular}
\label{fig:volr-examples}
\end{figure}

\FloatBarrier


\section{DSL implementation} \label{sec:implementation}
The 

% Duplication
In practice this happens through a replication of the previous layer, where
the output from the previous layer feeds into to both parallel 
layers ($e$ and $e'$).
In the other end, the two layers are merged such that each output neuron
is represented individually in a population with the size $e + e'$.
By not entangling the output dimensions, the following layers can choose
to ignore parts of the input, irrespective of the other parallel layer.
This parallelisation represents the duplication that occurs
in neural circuits, where structurally similar sub-networks, that are
fed the same stimuli, appear to contribute with semantically different
information (see \ref{sec:ref}).

% An optimal approach would be to find a tool that leverages the similarities
% of the network types, while integrating with the diverse simulated or emulated
% targets.
% That is, an abstract model of neural networks that can translate into
% heterogeneous back-ends, while retaining a high degree of inter-model validity.

% A domain-specific language (DSL) called Volr was recently presented to
% construct reproducible \gls{NN} experiments
% \autocite{Pedersen2018:volr}.

% Some work was required to fully support learning mechanisms on
% neuromorphic hardware, and the DSL, as well as the tooling around it, has been
% extended for the purpose of this thesis (see appendix \ref{appendix:volr})
% The following section describes the grammar and anatomy of Volr in detail.

% In practice a network is built by describing a graph.
% The nodes in the graph consist of \texttt{populations} of neurons and the edges
% are connection-set matrices to other populations \autocite{Djurfeldt2012}.
% % TODO: Describe CSA
% \texttt{Populations} can consist of any positive number of neurons and is
% required to have at least one connection.
% Connections can be recursive, resulting in a potentially cyclic graph.
% Both the connections and the \texttt{populations} can be annotated with features
% such as connection weight and neuron parameters (see \nameref{appendix:volr}).
% The parameters are treated differently depending on the experiment target (see
% sections \ref{sec:volr-NEST} and \ref{sec:volr-BrainScaleS}).

% \subsubsection{Experiment targets}
% The final element in a Volr experiment is its targets.
% A target describe a destination environments on which to run the experiment.
% These are described in detail in section \ref{sec:volr-targets}, and are
% referenced in the grammar as simple strings.

% \subsection{Volr semantics}


% \section{Neural network simulation targets in Volr} \label{sec:volr-targets}

% % TODO :Write how fields are interpreted
% % TODO: Write how input is interpreted

% Volr exploits the structural similarities between \gls{ANN} and \gls{SNN} to
% translate the model to both spiking and artificial network platforms (back-ends).

% In the remainder of the chapter the three emulation back-ends, shown in figure
% \ref{fig:volr}, are described:
% a machine learning target for \gls{ANN}s and a neuron simulation target, as well
% as a neuromorphic hardware target, for \gls{SNN}s.

% \begin{figure}
%   \centering
%   \includegraphics[width=0.6\textwidth]{images/volr-architecture.png}
%   \caption{The translation from the Volr DSL to \gls{ANN} simulations in OpenCL via
%     \gls{Futhark} and to \gls{SNN} simulations on \gls{NEST} and \gls{BrainScaleS}
%     via the \gls{Myelin} middleware.
%   }
%   \label{fig:volr}
% \end{figure}

% \subsection{Translation to Futhark} \label{sec:volr-futhark}
% Futhark is a functional data-parallel programming language \autocite{Henriksen2017}.
% It offers a number of compilation targets such as \gls{OpenCL}, which is
% particularly interesting for this thesis because of its capacity for hardware
% acceleration.

% The practical translation from the Volr model to Futhark is built on recurrent
% \gls{ANN} with stochastic gradient descent backpropagation
% \autocite{russel2007, schmidhuber2014}.
% Each neuron population is considered as a single layer, whose connections are
% determined by a connection matrix.

% % Deal with recurrent connections
% % Describe how this relates to layers

% ... To be continued ...

% \subsection{Spiking neural network simulations via PyNN} \label{sec:volr-pynn}
% The Python neural network simulation interface PyNN is designed as a
% "simulator-independent language for building neuronal network models"
% \autocite{PyNN2018}.
% It aims to reduce the problem of diverse, and occasionally unique, descriptions
% of neural network experiments for different simulation back-ends \autocite{Davison2009}.
% PyNN has been adapted by a number of simulators, including the NEST simulation
% platform and the neuromorphic BrainScaleS wafer system
% \autocite{Davison2009, Helias2012, Schmitt2017}.

% There are still simulator-dependent configurations that seems unlikely to be
% adopted into PyNN in the immediate future\footnote{
%   Particularly hardware mapping configurations are hard to abstract in a general
%   interface.
% }.
% For that reason Volr provides simulation-specific PyNN scripts that can
% interpret the model in the context of each simulation target.
% A middleware, dubbed \gls{Myelin}, was invented to translate the \gls{NN} model
% into a static intermediate representation in JSON.
% The JSON standard was chosen for the task because of its concise syntax while
% still retaining human readability.

% The advantage of the static experiment representation being, that the experiment
% easily a) transports to the target PyNN scripts without losing any information,
% and b) duplexes between several experiment; the same experiment setup is
% trivial to setup on multiple targets at once.

% The correct execution of the experiments relies on the PyNN scripts to exploit
% the simulator to represent the Volr model as accurately as possible.
% Fortunately PyNN is designed to cover exactly such a use case, so properties
% related to the \gls{NN} models itself (such as network topology and population
% attributes) were faithfully reproduced across the simulators.
% However, the simulators deviate in a number of ways that are relevant to
% mention.
% The following two sections explains the steps necessary to achieve accurate
% experiment environments in \gls{NEST} and \gls{BrainScaleS}.

% \subsubsection{Translation to PyNN} \label{sec:volr-translation}

% \subsubsection{Translation to NEST} \label{sec:volr-NEST}
% ... To be continued ...
% \subsubsection{Translation to BrainScaleS} \label{sec:volr-BrainScaleS}
% ... To be continued ...
\label{ref:Cairo}


\section{DSL verification} \label{sec:verification}
To verify that the DSL implementation was successful, and that the models perform as expected when evaluated on the
backends, a number of tests were written and automated.
This sections elaborates on the reasoning and design of the tests, which are divided into two categories: unit
tests and integration tests.
\\[0.2cm]
All tests are available online, see Appendix \ref{app:implementation}.

\subsection{Unit tests}
\subsubsection{Volr compiler}
Each expression construct in the compiler---and their combinations---are
tested as to whether the expected output is produced, such that the evaluator
is guaranteed to generate well-formed code.
An example is shown in Listing \ref{lst:eval-test},
in which a unit test \index{unit test} verifies that the let
binding of the constant \texttt{x} correctly
evaluates to the network \texttt{(\textbf{dense} 1 1)}.

\begin{minipage}{\linewidth}
\begin{lstlisting}[language=haskell,caption={Part of the evaluation code in
Haskell.},label={code:evaluator}]
eval' :: Term -> EvalState Term
eval' term =
  case term of
    TmNet n m -> return $ TmNet n m
    TmSeq t1 t2 -> do
      t1' <- eval' t1 
      t2' <- eval' t2
      return $ TmSeq t1' t2'
\end{lstlisting}
\end{minipage}

\begin{lstlisting}[language=Haskell,label={lst:eval-test},caption={A unit test for the correct evaluation of a let binding.}]
it "can evaluate a let binding with a reference" $ do
  let e = TmLet "x" (TmNet 1 1) (TmRef "x")
  eval e `shouldBe` Right (TmNet 1 1)
\end{lstlisting} 

\subsubsection{Futhark backend}
The Futhark backend was tested using unit tests (using \texttt{futhark-test}
\cite{Elsman2018}) for each layer, activation function and loss function.
Tests for the dense layer already existed in the library used (\cite{Minh2018}). Tests for the merge and
replicate layers were added using manual calculations of the expected gradients, as shown in Listing
\ref{lst:futhark-replicate-test}.The input should produce the expected weight
gradients (\texttt{output}) here.
Further tests for the different combinations of parallel layers were also added.

\begin{minipage}{\linewidth}
\begin{lstlisting}[language=,label={lst:futhark-replicate-test},caption={A test for the
correct calculation of the backwards weight gradient during backpropagation in a
replicate layer}]
-- ==
-- entry: replicate_backward_w
-- input {[[1.0, 2.0, 3.0, 4.0],
--         [2.0, 3.0, 4.0, 5.0],
--         [3.0, 4.0, 5.0, 6.0]]
--
--        [[1.0,  2.0,  3.0,  4.0],
--         [5.0,  6.0,  7.0,  8.0],
--         [9.0, 10.0, 11.0, 12.0]]
--
--         [1.0, 2.0, 3.0]}
--
-- output {[[-25.60,  -36.90,  -48.20,  -59.50],
--          [-59.00,  -87.40, -115.80, -144.20],
--          [-92.40, -137.90, -183.40, -228.90]]}

entry replicate_backward_w input w b =
  let ws = ((w, b), (w, b))
  let (cache, output) = replicate.forward true ws input
  let (_, ((w',_), (_, _))) = replicate.backward false updater ws cache output
  in w'
\end{lstlisting}
\end{minipage}

\subsubsection{PyNN}
The backend-agnostic code of PyNN was tested using the \texttt{Pytest} framework
\cite{Pytest2018}.
This includes test for the activation functions, spike normalisation functions, error functions, 
as well as general Python structure.

Especially the model class in the PyNN backend contains error-prone code, because it
deals with stateful backends.
Unit tests are therefore particularly crucial.
The test shown in Listing \ref{lst:pynn-model-test} shows an example of such a
test that, in this case, validates that the simulator is properly reset between runs:

\begin{lstlisting}[language=Python,label={lst:pynn-model-test},caption={Unit test for
PyNN model to validate that the model correctly updates weights}]
def test_nest_model_backwards_reset():
    p1 = pynn.Population(2, pynn.IF_cond_exp())
    p2 = pynn.Population(2, pynn.IF_cond_exp())
    l1 = v.Dense(p1, p2, v.ReLU(), decoder = v.spike_count_normalised, weights = 1)
    m = v.Model(l1)
    xs1 = np.array([1, 1])
    ys1 = np.array([0, 1])
    xs2 = np.array([1, 1])
    ys2 = np.array([0, 1])
    # First pass
    target1 = m.predict(xs1, 50)
    m.backward([0, 1], lambda w, g, b, bg: (w - g, b - bg))
    expected_weights = np.array([[1, -1], [1, -1]])
    assert np.allclose(l1.get_weights(), expected_weights)
    # Second pass
    target2 = m.predict(xs2, 50)
    m.backward([1, 0], lambda w, g, b, bg: (w - g, b - bg))
    expected_weights = np.array([[-1, -1], [-1, -1]])
    assert np.allclose(l1.get_weights(), expected_weights)
\end{lstlisting}

The PyNN code for the NEST backend was also tested with \texttt{Pytest}
\cite{Pytest2018}.
A unit test for the backpropagation was written using numerical gradient descent
with a simulated feedforward step.
In the test, the layer weights are slightly skewed to approximate the movement
along a gradient, which, in the test, was based on a sigmoid function.
The resulting changes in the backpropagated error should be minuscule, provided
that the algorithm has been implemented correctly.
A snippet of the test is shown below.

\begin{minipage}{\linewidth}
\begin{lstlisting}[language=Python,label={lst:volrpynn_numerical},caption={Part
of the numerical gradient test for the densely connected layer in PyNN.}]
# Calculate numerical gradients
numerical_gradients = compute_numerical_gradient(xs, ys)
# Calculate 'normal' gradients
gradients = compute_gradients(xs, ys)
# Calculate the ratio between the difference and the sum of vector norms
ratio = np.linalg.norm(gradients - numerical_gradients) /\
           np.linalg.norm(gradients + numerical_gradients)
assert ratio < 1e-07
\end{lstlisting}
\end{minipage}

\subsubsection{Continuous integration}
Continuous integration (CI) is a tool to automatically trigger the unit tests, and is
typically associated with updates in a version control system. \index{Continuous
integration}
The projects in this thesis are all exploiting this to continually verify that
the software does not regress.
Whenever changes are committed and published to their respective repositories on
GitHub, the unit tests are executed.
Unit tests for NEST are, however too computationally expensive for the
continuous integration service due and are left out.

\subsection{Integration tests}
Integration tests exists to verify that the entire toolchain is functioning.
It is not intended to test the correctness of the individual parts, but rather
that they correctly integrate with each other.
The tests are performed inside a containerised environment, as a
method to ensure a homogeneous environment. 

Integration tests have been performed during the construction of the software,
but no automated tests are in place.

The tests assert that the \gls{DSL} compiler is capable of compiling the models
into code for the individual backends that executes and provides the correctly
formatted result.
The actual values are not verified, because the integration tests assume that
the projects are well-behaved independently.


\end{document}
