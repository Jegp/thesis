This sections presents the \gls{DSL} Volr. \index{Volr}
The main purpose of Volr is to define clear and reproducible
experiments whose semantics are retained regardless of
the runtime environment.
The specification in its current form is relatively simple, but sufficiently 
complicated for the purpose of this thesis.
It focuses solely on the topology of networks, thus
separating the network description from any generation-specific properties
of neurons or neuron populations.

The first requirement is met through an unambiguous syntax heavily inspired
by the lamdba calculus \cite{Pierce2002}.
Figure \ref{fig:volr-expr} shows the BNF notation for expressions, values and types
in Volr. 
Figure \ref{fig:volr-rules} lists evaluation rules for the correct
interpretation of the expressions.

% Expression figure
\begin{figure}
  \begin{tabular}[t]{l l}
    expressions & \texttt{$e$ ::= $n$} \\
    & \begin{minipage}{0.6\textwidth}
      \begin{Verbatim}[mathescape,commandchars=\\\{\}]
    | \textbf{dense} $n\ m$
    | \textbf{let} $x = e$ \textbf{in} $e'$
    | $e\ \obar\ e'$
    | $e\ \ominus\ e'$
    | $\neg e$
      \end{Verbatim} 
      \end{minipage} \\

    & \\ % Empty space 

    values
    & \texttt{$v$ ::= $\textbf{net}\ n\ m$} \\
    
    & \\ % Empty space
    types
    & \texttt{$\tau$ ::= \textbf{int} | \textbf{net} $n\ m$} \\
  \end{tabular}

  \caption{Expressions, values and types of the Volr language.}
  \label{fig:volr-expr}
\end{figure}
\begin{figure}
\begin{prooftree}
  \AxiomC{}
  \UnaryInfC{$\Gamma \vdash n : \mathbf{int}$}
\end{prooftree}
\begin{prooftree}
  \AxiomC{}
  \UnaryInfC{$\Gamma \vdash \mathbf{stim}\ n : \mathbf{layer}\ n$}
\end{prooftree}
\begin{prooftree}
  \AxiomC{}
  \UnaryInfC{$\Gamma \vdash \mathbf{pop}\ n : \mathbf{layer}\ n$}
\end{prooftree}
\begin{prooftree}
  \AxiomC{$\Gamma (x) = \tau$}
  \UnaryInfC{$\Gamma \vdash x : \tau$}
\end{prooftree}
\begin{prooftree}
  \AxiomC{$\Gamma \vdash e_1 : \mathbf{layer}\ n$}
  \AxiomC{$\Gamma \vdash e_2 : \mathbf{layer}\ m$}
  \AxiomC{$\Gamma \vdash w : \mathbf{float}$}
  \TrinaryInfC{$\Gamma \vdash \otimes\ e_1\ e_2\ w : \mathbf{con}\ n\ m$}
\end{prooftree}
\begin{prooftree}
  \AxiomC{$\Gamma \vdash e_1 : \mathbf{layer}\ n$}
  \AxiomC{$\Gamma \vdash e_2 : \mathbf{layer}\ m$}
  \AxiomC{$\Gamma \vdash w : \mathbf{float}$}
  \TrinaryInfC{$\Gamma \vdash \ominus\ e_1\ e_2\ w : \mathbf{con}\ n\ m$}
\end{prooftree}
\begin{prooftree}
  \AxiomC{$\Gamma \vdash e : \tau$}
  \AxiomC{$\Gamma [v : \tau] \vdash e' : \tau$}
  \BinaryInfC{$\Gamma \vdash \mathbf{let}\ x = e\ in\ e' : \tau$}
\end{prooftree}

  \caption{Evaluation rules in Volr.}
  \label{fig:volr-rules}
\end{figure}



The constant expression $n$ is simply an integer that evaluates to the type 
\texttt{\textbf{int}} ($e1$). 
Similarly to the lambda calculus, the \texttt{\textbf{let}} binding binds
the string constant $x$ to the expression $e$ when evaluating $e'$ \cite{Pierce2002}.
That constant can later be referenced in the $e'$ expression 
through the string $x$ as shown in $e2$.

The \texttt{\textbf{dense}} expression describes a fully connected
neural network layer with a single population, and is the most basic
concept in the \gls{DSL}.
Notice the distinction between a neural network \textit{layer}
such as \texttt{\textbf{dense}} and a population. 
A \texttt{\textbf{dense}} network layer can be understood as a single
population that receives a number of inputs, densely 
connects these inputs to all neurons in the population with the associated
weights, and finally outputs the response to the following layer.
Considering how each neuron is a type of classifier, $n$ and $m$
illustrate the \textit{dimensionality} of the network, such that the number of
dimensions in the input is truncated (or expanded) to classify the
dimensionality of the output layer.

The $\ominus$ (parallel) operator parallelises two networks by duplicating
the input from the previous layer and merging the outputs into a single
layer ($e5$).
The input feeds into both $e$ and $e'$, such that the input dimension of
the network must be shared by the two layers ($l$). 
The output from the network is stacked such that each neuron from each
population corresponds to one output neuron ($e_{out} + e'_{out}$).
This is done to preserve the meaning of each parallel population.

The $\obar$ (sequential) operator binds two networks sequentially,
such that the output layer of the first network becomes the input layer 
of the second network.
For that reason the output dimensionality of the first network ($m$) should 
be equal to the input of the second network, as shown in $e4$.

Semantically the parallel operator provides the ability for a network to
perform specialised functions, based on the same stimulus.
In the context of neural systems such specialisations are frequently used
to balance correctness with compactness: it is cheaper in terms of neurons,
and more efficient in terms of accuracy,
to allow two subnetworks to specialise than to have one large network 
generalise.

% TODO: Relate this to the trade-off wrt. training in the theory section
\begin{comment} %TODO: include this?
In the other end, the two layers are merged such that each output neuron
is represented individually in a population with the size $e + e'$.
By not entangling the output dimensions, the following layers can choose
to ignore parts of the input, irrespective of the other parallel layer.
This parallelisation represents the duplication that occurs
in neural circuits, where structurally similar sub-networks, that are
fed the same stimuli, appear to contribute with semantically different
information (see \ref{sec:ref}).
\end{comment}

Taken together these constructs can express simple neural networks and
the properties of their connections. 
Figure \ref{fig:volr-examples} shows a number of example networks
that visualises four examples of networks. 

\begin{figure}
  \ContinuedFloat*
  \begin{tabular}[t]{c c}
    \begin{minipage}{0.5\textwidth}
      \begin{Verbatim}[mathescape,commandchars=\\\{\}]
\textbf{let} s = stim 2 \textbf{in}
  \textbf{let} p = pop 2 1 \textbf{in}
    s $\otimes$ p
      \end{Verbatim}
    \end{minipage} & \begin{minipage}{0.5\textwidth}
      \includegraphics[width=\textwidth]{chapters/volr/example1.pdf}
    \end{minipage}

  \end{tabular}
  \caption{A textual and visual example of a network with a stimulus,
    a population and an implicit output.}
  \label{fig:volr-example1}
\end{figure}

\begin{figure}
  \ContinuedFloat*
  \begin{tabular}[t]{c c}
    \begin{minipage}{0.5\textwidth}
      \begin{Verbatim}[mathescape,commandchars=\\\{\}]
\textbf{let} s = stim 2 \textbf{in}
  \textbf{let} p = pop 2 1 \textbf{in}
    s $\otimes$ p
      \end{Verbatim}
    \end{minipage} & \begin{minipage}{0.5\textwidth}
      Hello
    \end{minipage}
  \end{tabular}
\label{fig:volr-examples}
\end{figure}

\FloatBarrier
