\documentclass[report.tex]{subfiles}
\begin{document}

The field of \gls{ml} is rapidly evolving, and have in
some recognition tasks surpassed human-level precision
\autocite{Schmidhuber2014}.
This acceleration is propelled by the advances in \gls{ANN}s
\autocite{Rumelhart1988, Schmidhuber2014, Nilsson2009}, or
so-called \textit{second} generation \gls{ANN}s\footnote{Not to be 
understood as more advanced than previous generations, see definition on page
\pageref{sec:glossary}.} \cite{Maass1997}.
While the first generation networks build on the \textit{McCulluch-Pitts}
neuron model that only produces binary outputs (0 or 1), the second
generation uses a sequentiasl layered approach with a continuous set of 
possible outcomes \cite{Maass1997, Russel2007}.
Contemporary cognitive neuroscience however, tells us that the computation
in the biological brain uses temporally separated \textit{spikes}, 
dispatched within a highly connected and recurrent architecture 
\cite{Dayan2001, Eliasmith2004}.
This \textit{third} generation of \gls{NN}s permit the encoding
of information over time, but at the cost of stability and 
noise \autocite{Maass1997}.

Because of their biological similarities, third generation 
networks are of great interest to (cognitive) neuroscientists
\autocite{Dayan2001,Bruderle2011,Eliasmith2015}.
Compared to experimental studies involving living neural substrate,
it is significantly cheaper and faster to build neural 
models either as pure simulations
\autocite{Davison2009, Eliasmith2015} or as analogue circuits
that resemble the physical structure of neural networks
\autocite{Walter2015, Schmitt2017}.
Furthermore, researchers have complete control over virtual models
to pause, lesion, or even disassembled at will.

This is particularly interesting for the
mapping of functional connectivity (the human connectome) to the
processes of cognition \autocite{Eliasmith2015, Mogensen2011}. 
\textcite{Mogensen2011} presented a model for the 
\gls{REF} based on numerous clinical trials and lesion studies.
However, the theory is challenging to verify in vivo because of
the high level of interconnectivity in the mammalian brain
\autocite{Hohwy2009, Mogensen2017}.

A challenge for third generation networks is the relatively
poor understanding of learning mechanisms \autocite{Tavanei2015,
Walter2015}.
Several different techniques have been applied, but they are
difficult to validate because of the heterogeneous environments and
the incommensurability of the models \autocite{Bohte2002, Bruderle2011}.

This thesis sets out to create a unified domain specific language (DSL), Volr, 
for modelling both second and third generation neural networks.
The DSL aims to provide a concise description of neural networks and to 
ensure the correct translation into executable neural network models.
This thesis will implement translations into three heterogenous backends: 
\gls{OpenCL} for non-spiking second generation \gls{NN}s, NEST for simulated third
generation \gls{NN}s, and finally
BrainScaleS for analogue third generation \gls{NN}s.

% TODO: Add section describing the backends

To validate the DSL, a cognitive model inspired by the theory for the Reorganisation
of Elementary Functions [8] is constructed and applied to solve a small MNIST classification task.
The present study aims to extend the innovations within
second and third generation \gls{NN}s to the field of
cognitive \index{cognitive} neuroscience. 
Constructing a \gls{DSL} as a means to translate higher-order
neural concepts into \gls{NN} simulations, allows for a higher
degree of verification and permits rigorous analyses of the networks.
For concepts such as those found in the \gls{REF} theory, 
this can lead to a further understanding, verification and falscification
of simple neural models.

%TODO: Add section on problem with state of the art
\section{Hypothesis} \label{ref:hypothesis}
This thesis examines two hypotheses:

\begin{enumerate}
  \item The Volr DSL can translate into spiking and non-spiking neural networks such that the network topologies are retained.
  \item Using training it is possible for spiking and non-spiking models to solve an MNIST recognition task.
\end{enumerate}

Topology is understood as typed declarations of neural network structure, such
that the topology of a network is defined as a collection of nodes, edges and connectivity descriptions.

The first hypothesis tests that the neural networks generated by the DSL is translated without significant
deviations, regardless of the backend.
Consistent translations are important to ensure correct and reproducible experiments,
but are also vital to further the understanding of spiking neural networks:
correct rendition bridges the semantics of artificial and spiking neural networks.
In turn, this allows users of the DSL to draw on the vast literature of
second generation \gls{NN}s.

The second hypothesis tests that the DSL adequately models the domain it sets
out to describe: cognitive neural networks.
By drawing on the REF theory, a network model is written, trained, and evaluated to verify
that the DSL can capture complex neural network concepts.

Three \textit{implementation}s of the \gls{REF} model will be
produced to test the hypothesis: 
  1) a second generation \gls{NN} based on the data-parallel language Futhark, \index{Futhark}
  2) a third generation \gls{NN} based on the neural simulator \gls{NEST} and
  3) a third generation \gls{NN} based on the neuromorphic \index{neuromorphic}
     hardware platform BrainScaleS \index{BrainScaleS}.

The above models will be compared based on:
1) the ability to learn the maze problem as well as 
2) the proximity of the neural model to the components of the
   \gls{REF} theory.

Regarding 1) it is known that second generation \gls{NN}s are
capable of learning pattern-recognition problems (see Section \ref{sec:learning})
\cite{Russel2007}.
This adaptation is a fundamental property of neural systems, so a
similar behaviour is expected in third generation \gls{NN}s.
If that is not the case, the \gls{DSL} has failed to translate
the learning capacities of the second generation networks,
disproving the hypothesis.

Specifically a third generation \gls{NN} is expected to
achieve a precision within 10\% of a topologically similar
second generation \gls{NN}.

Regarding 2) the presence of significant
dissimilarities\footnote{See \ref{sec:ref} for a definition of
\enquote{dissimilar}.} between the neural network implementations 
and the abstract \gls{REF} model also falsifies the hypothesis. 
In that case the \gls{DSL} is unable to accurately express the
theoretical model.

Specifically, it is expected that the generated \gls{NN}s resemble
the structures from the \gls{REF} theory.

\section{Scope} \label{sec:scope}
The \gls{REF} model is a general and abstract blueprint of cognitive
processes, and no attempts have as of yet been made to map it to
a computational model.
This first attempt is only made feasible by focusing on select parts
of the \gls{REF} model, explained in Section \ref{sec:ref}.
It is outside the scope of this thesis to examine
higher-order cognitive functions as well as detailed biological
similarities.

While this thesis aims to build models that shows some of the same
properties that are present in the biological brain,  
it is not the goal to provide a mapping between the
implementations and their neurophysiological equivalent.
Neither is it a goal to translate the \gls{DSL} into exact replicas
of second and third generation networks, for the reason that the
two modelling paradigms are vastly different.\footnote{See \ref{sec:nn}
for a detailed description of the dissimilarities.}
This will naturally have consequences for the subsequent analysis of the
experimental results (see Section \ref{sec:discussion}).

\section{Thesis structure}
Before the hypothesis can be formally tested, Chapter \ref{sec:theory}
presents the foundation for \gls{NN}s, the \gls{REF} theory and
concepts of learning.
The translation of the constituent parts of the \gls{REF} theory
into \gls{NN} components is presented in Chapter \ref{sec:volr} in 
the form of the \gls{DSL} \textit{Volr}. \index{Volr}
The fourth Chapter defines an \gls{NN} model built with the above
\gls{DSL} as well as a delimited T-maze experiment, on which the 
\gls{NN} model will be trained.
This Chapter also presents the three \gls{backend}s on which the 
model will be trained, as well as the accuracy scores of the models
before and after the alterations.
Chapter \ref{sec:analysis} analyses the test scores and the third 
generation models themselves to measure how well they capture the
concepts of the \gls{REF} model.

In the fifth and final part of the thesis, the analysis and the benchmark
results will be discussed and related to the hypothesis.
This part also concludes on the findings of the thesis and
reflects on the prospects for employing Volr to study neurocognitive
models in the future.

% TODO: Write contributions

\end{document}
