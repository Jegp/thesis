
\documentclass[report.tex]{subfiles}
\begin{document}

Building on research from cognitive and computational neuroscience, deep
learning is evolving rapidly and has even surpassed humans in some
recognition tasks \cite{Schmidhuber2015}.
Contemporary theories from cognitive neuroscience however, tell us
that learning in the biological brain occurs in spiking neural networks
instead of the layered neural networks used in machine learning
\cite{Dayan2001}.

\textcite{Pfeil2013}, , Tavanaei and Maida [4] and Walter et al. [5] have
shown that spiking neural networks are capable of solving a wide range
of learning tasks.
The neuromorphic hardware platform BrainScaleS has even been showed to
learn classification tasks [6]. Such platforms however, require
considerable configuration in both hardware and software [5, 6, 7].

Because of their biological similarities neuromorphic hardware is of
great interest to neuroscientists [7].
The programming and configuration needed to setup neuromorphic
experiments are however inaccessible to most neuroscientists.

Based on a Krechevky maze-learning task [8] and the theory of Reorganisation of Elementary Funtions (REF) [9], this thesis sets out to explore how a more accessible domain specific language (Volr) [10] can help to build and evaluate experiments in neuromorphic hardware.
By building on the REF model, the setup can be transferred to other learning tasks, if successful, and could lay the foundation for a robust neuromorphic experimental framework for cognitive neuroscientists.

\section{Hypothesis}
This thesis examines the hypothesis that *the model for the Reorganisation of Elementary Functions can be implemented using spiking neural networks*.

The hypothesis drives two outcomes: a spiking neural network representation of the REF model and a Krechevky maze experiment.

\subsection{Scope}
The hypothesis will be evaluated based on two criteria: the similarity of the spiking REF model to contemporary neurocognitive theories on the mammalian brain as well as its capacity to learn a given problem.

The thesis will focus on how well the the model describes contemporary neurocognitive theories of learning, keeping the limits of the experimental platforms in mind.
Because the REF model has not been mapped to its physiological properties, it is outside the scope of this thesis to provide an exact comparison to the biological properties.

To provide a context for the learning capacity of the spiking neural networks, they will be compared to a regular non-spiking neural network.

\section{Experimental setup}
Three models will be built: a spiking neural network simulation via NEST [11], a spiking neural network emulation via BrainScaleS [6] and a layered non-spiking neural network written in Futhark [12].
All three models will be trained to perform the same Krechevky maze task.

Since the neuromorphic hardware is significantly more performant than the software simulation, it is desirable to run it on chip.
It is however, important to retain the simulation as a baseline, because of unexpected analogue effects of the neuromorphic hardware.

The non-spiking neural network will be written in Futhark and executed using OpenCL.
\section{Cognitive neuroscience}
\section{Neuromorphic hardware}
% Motivation: platform for simulation
\section{Hypothesis} \label{sec:hypothesis}
This thesis examines the hypothesis that *the model for the Reorganisation of Elementary Functions can be implemented using spiking neural networks*.
\subsection{Evaluation criteria} \label{sec:hypothesis-criteria}


\end{document}