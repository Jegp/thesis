\documentclass[report.tex]{subfiles}
\begin{document}

The field of machine learning is rapidly evolving, and have in
some recognition tasks surpassed human-level precision
\autocite{Schmidhuber2014}.
This acceleration is propelled by the advanced in artificial neural
networks \autocite{Rumelhart1988, Schmidhuber2014, Nilsson2009}, or
so-called \textit{second} generation networks \cite{Maass1997}.
The architecture behind these networks developed from the neuroscientific
idea of a neuron \autocite{Nilsson2009, Russel2007}.
Contemporary cognitive neuroscience however, tells us that learning in
the biological brain occurs in spiking neural networks, where neurons
communicates in \textit{spikes} over time \cite{Dayan2001, Eliasmith2004}.
This \textit{third} generation of neural networks permits the encoding
of information over time, but at the cost of stability and 
noise \autocite{Maass1997}.

Because of their biological similarities, this third generation of
networks are of great interest to neuroscientists
\autocite{Dayan2001,Bruderle2011,Eliasmith2015}.
Compared to experimental studies involving living neural substrate,
it is significantly cheaper and faster to build cognitive 
models \autocite{Eliasmith2015}; either as pure simulations
\autocite{Davison2009} or as analogue circuits
that resemble the physical structure of neural networks
\autocite{Walter2015, Schmitt2017}.

% One example where simulations could support the research efforts, is
% the mapping of functional connectivity (the connectome) to the processes
% of cognition \autocite{Eliasmith2015, Mogensen2011}. 
% \textcite{Mogensen2017} presented a model for the 
% \textit{reorganisation of elementary functions} (REF) as an attempt
% to relating conscious states to neural matter.


% It is unclear whether the advances in layered neural networks can 
% reproduce the complicated mechanisms for learning and cognition,
% brought about in neurobiological systems \autocite{Eliasmith2015, Pearl2018}.

% \textcite{Eliasmith2015} have implemented von Neumann style computation 
% in neurobiological systems, and a number of studies
% have shown that spiking neural networks are capable of solving a wide
% range of learning tasks \autocite{Pfeil2013, Tavanei2015, Walter2015},
% proving that their .
Acknowledging the importance of accessible and realistic \textit{third} 
generation neural simulations for the cognitive sciences, the first
part of this thesis
sets out to build a \gls{DSL} that prototypes the translation of
abstract neural networks into computational structures that
can be evaluated either as second or third generation neural networks.

To validate the applicability of the \gls{DSL} the second part of 
the thesis focuses on the modelling of a neural network that follows
the principles of the cognitive theory by \textcite{Mogensen2011} of
reorganising elementary functions (\gls{REF}).
The \gls{REF} model guides the description of a neural network
that is capable of solving a simple Krechevsky maze task 
\autocite{Krechevsky1932}.

The model will be evaluated based on two parameters: 
its ability to  learn the maze task and its relevance to the 
\gls{REF} model (see section \ref{sec:ref}).
This evaluation will happen in both simulated and analogue
third generation neural network environments, building on the
work of particularly two contemporary third generation simulation
frameworks: NEST and PyNN \autocite{Davison2009}.
To provide a means of comparing the quality of the spiking models,
a second generation neural network will be built using the
data-parallel functional language Futhark \autocite{Henriksen2017}.

\section{Hypothesis}
This thesis examines the hypothesis that *the model for the Reorganisation
of Elementary Functions can be implemented using spiking neural networks*.

The hypothesis drives two outcomes: a spiking neural network
representation of the REF model and a Krechevky maze experiment.

\subsection{Scope}
The hypothesis will be evaluated based on two criteria: the similarity of
the spiking REF model to contemporary neurocognitive theories on the
mammalian brain as well as its capacity to learn a given problem.

The thesis will focus on how well the the model describes contemporary
neurocognitive theories of learning, keeping the limits of the
experimental platforms in mind.
Because the REF model has not been mapped to its physiological
properties, it is outside the scope of this thesis to provide an exact
comparison to the biological properties.

To provide a context for the learning capacity of the spiking neural networks, they will be compared to a regular non-spiking neural network.

\section{Experimental setup}
Three models will be built: a spiking neural network simulation via NEST [11], a spiking neural network emulation via BrainScaleS [6] and a layered non-spiking neural network written in Futhark [12].
All three models will be trained to perform the same Krechevky maze task.

Since the neuromorphic hardware is significantly more performant than the software simulation, it is desirable to run it on chip.
It is however, important to retain the simulation as a baseline, because of unexpected analogue effects of the neuromorphic hardware.

The non-spiking neural network will be written in Futhark and executed using OpenCL.
\section{Cognitive neuroscience}
\section{Neuromorphic hardware}
% Motivation: platform for simulation
\section{Hypothesis} \label{sec:hypothesis}
This thesis examines the hypothesis that *the model for the Reorganisation of Elementary Functions can be implemented using spiking neural networks*.
\subsection{Evaluation criteria} \label{sec:hypothesis-criteria}


\end{document}