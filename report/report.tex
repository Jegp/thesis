\documentclass[a4paper,oneside]{memoir}
% Divide report into subfiles
\usepackage{subfiles}

\usepackage[english]{babel}
\usepackage[margin=1cm]{caption}
\usepackage[T1]{fontenc}
\usepackage[utf8]{inputenc}
\usepackage{wallpaper}
\usepackage{palatino}
\usepackage{hyperref}
\usepackage{csquotes}
\usepackage{fvextra}
\usepackage{xpatch}
\usepackage{amsmath}

% tikz drawing
\usepackage{tikz}
\usetikzlibrary{shapes,arrows,matrix,chains,positioning,decorations.pathreplacing,arrows}

% Listings
\usepackage{listings}
\lstset{
  basicstyle=\ttfamily,
  mathescape
}

% Proofs
\usepackage{bussproofs}

% Floats and figures
\usepackage{float}
\usepackage{wrapfig}
\restylefloat{figure}
\usepackage{caption}
\DeclareCaptionLabelFormat{cont}{#1~#2\alph{ContinuedFloat}}
\captionsetup[ContinuedFloat]{labelformat=cont}

%linespacing
\usepackage{setspace}
\renewcommand{\baselinestretch}{1.5}

% Grammar
\usepackage{fancyvrb}

% Internal references
\usepackage{cleveref}

% Bibliography
\usepackage[style=authoryear]{biblatex}
\addbibresource{bib.bib}
\bibliography{bib}

% Promote sections and subsections
\setheadfoot{\onelineskip}{2\onelineskip}
\setheaderspaces{*}{1mm}{*}

% Glossary
\usepackage[numberedsection=nameref]{glossaries}
\renewcommand{\glossarypreamble}{\label{glos}}
\makeglossaries
\newglossaryentry{ai} {
    name = artificial intelligence,
    description = {Artificial intelligence (AI) covers the broad discipline in computer science
that is concerned with replicating intelligent behaviour in computational systems. The exact
definition is controversial for historical reasons \autocite{Nilsson2009}}
}
\newglossaryentry{ANN} {
  name = artificial neural network,
  description = {Artificial neural networks...}
}
\newglossaryentry{BrainScaleS} {
  name = BrainScaleS,
  description = {...}
}
\newglossaryentry{computation} {
   name = computation,
   description = {Computation refers to any process (in any
substrate) that can deduce new information based on old information. In
this is manifested as computing instructions}
}
\newglossaryentry{DNN} {
  name = {deep neural network},
  description = {Deep neural networks...}
}
\newglossaryentry{dsl} {
  name = domain specific language,
  description = {A DSL is a language used to model concepts from a certain
    domain. DSLs are usually simpler than more general programming languages in
    that they contain fewer concepts and less complex syntax}
}
\newglossaryentry{Futhark} {
   name = {Futhark},
   description = {A programming language geared towards performance in parallel environment such as
   graphics processors (GPUs). Futhark is a purely functional array language and is
   developed by HIPERFIT research center under the Department of Computer Science at the
   University of Copenhagen (DIKU)}
}
\newglossaryentry{NEST} {
  name = NEST,
  description = {...}
}
\newglossaryentry{NN} {
  name = {neural network},
  description = {...} % TODO
}
\newglossaryentry{ncc} {
   name = {NCC},
   description = {Neural patterns or condition that is minimally sufficient for a conscious
thought to occur. See \autocite{atkinson2000, Hohwy2009}}
}
\newglossaryentry{ml} {
  name = machine learning,
  description = {Machine learning is a sub-field within \gls{ai} that is concerned
    with developing systems that "progressively improves their performance on a
    certain task" \autocite{wiki:ml}}
}
\newglossaryentry{meme} {
name = meme,
description = {\textit{Meme} is a shortened form of the ancient Greek \textit{mimeme} meaning
'imitated thing' and was coined by Richard Dawkins. A meme refers to a idea or a
\textit{way of behaving} that can be \enquote{copied, transmitted, remembered, taught, shunned,
brandished, ridiculed, parodied, censored, hallowed} \autocite{dennett2017}}
}
\newglossaryentry{Myelin} {
  name = Myelin,
  description = {...}
}
\newglossaryentry{OpenCL} {
   name = {OpenCL},
   description = {An open standard for cross-platform parallel programming, which
   allows software to be executed on CPUs, GPUs or other processors or hardware accelerators. See \url{
   https://www.khronos.org/opencl/}}
}
\newglossaryentry{REF} {
  name = REF,
  description = {A model for rehabilitation in patients with brain lesions, developed
    by \cite{Mogensen2011}. An extension in the form of the REFGEN model was developed by
    \cite{Mogensen2017}}
}
\newglossaryentry{SNN} {
  name = {spiking neural network},
  description = {Spiking neural networks...}
}

\makeglossaries

% TOC
\maxsecnumdepth{subsection}

% \raggedbottom

% Chapter settings
\chapterstyle{ger}

% Change memoir chapter margins
\usepackage{titlesec}
\titleformat{\chapter} % command
  [display] % shape
  {\normalfont\huge\bfseries} % format
  {\chaptertitlename\ \thechapter} % label
  {5pt} % Separator
  {
    % \rule{\textwidth}{1pt}
    % \vspace{1ex}
    % \centering
  } % Before code
  [
    \vspace{-0.5ex}
    \rule{\textwidth}{0.3pt}
  ] % after code

% \titlespacing*{\chapter}{10pt}{10pt}{10pt}

%%  Setup fancy style quotation
%%  ==================================================================
%\usepackage{tikz}
%\usepackage{framed}

%\newcommand*\quotefont{\fontfamily{fxl}} % selects Libertine for quote font

% Make commands for the quotes
%\newcommand*{\openquote}{\tikz[remember picture,overlay,xshift=-15pt,yshift=-10pt]
%     \node (OQ) {\quotefont\fontsize{60}{60}\selectfont``};\kern0pt}
%\newcommand*{\closequote}{\tikz[remember picture,overlay,xshift=15pt,yshift=5pt]
%     \node (CQ) {\quotefont\fontsize{60}{60}\selectfont''};}

% select a colour for the shading
%\definecolor{shadecolor}{rgb}{1,1,1}

% wrap everything in its own environment
%\newenvironment{shadequote}%
%{\begin{snugshade}\begin{quote}\openquote}
%{\hfill\closequote\end{quote}\end{snugshade}}

%%  Begin document
%%  ==================================================================
\begin{document}

%%  Begin title page
%%  ==================================================================
    \thispagestyle{empty}
    \ULCornerWallPaper{1}{ku-coverpage/nat-farve.pdf}
    \ULCornerWallPaper{1}{ku-coverpage/diku-en.pdf}
    \begin{adjustwidth}{-3cm}{-1.5cm}
    \vspace*{1cm}
    \textbf{\Huge Modelling neural learning systems} \\
    \\
    \vspace*{1cm}
    \hskip-2pt
    {\huge Describing learning tasks in spiking and non-spiking networks }\\
    \begin{tabbing}
    % adjust the hspace below for the longest author name
    Jens Egholm Pedersen \hspace{1cm} \= \texttt{<xtp778@alumni.ku.dk>} \\
    \\[11cm]

    \textbf{\Large Supervisor} \\
    Martin Elsman \hspace{1cm} \texttt{<mael@di.ku.dk>}
    \end{tabbing}
    \end{adjustwidth}

    \newpage

    \ClearWallPaper
%%  ==================================================================
%%  End title page

\renewcommand\cftchapteraftersnumb{\normalfont}
\renewcommand\cftbeforechapterskip{5pt plus 1pt}

\frontmatter
\setcounter{tocdepth}{2}
\tableofcontents*
\newpage

\mainmatter
\chapter{Introduction} \label{sec:intro}
  \subfile{chapters/introduction}

\chapter{Theory}
  \subfile{chapters/theory}

\chapter{Neural network modelling in Volr} \label{sec:volr}
  \subfile{chapters/volr}

\chapter{Neural network architecture} \label{sec:model}
  \subfile{chapters/model}

\chapter{Experimental setup} \label{sec:experiment}
  \subfile{chapters/experiment}

\chapter{Data collection}
  \section{Futhark}
  \section{NEST}
  \section{BrainScaleS}

\chapter{Analysis}
  \section{Experiment results}
  \subsection{Futhark}
  \subsection{Nest}
  \subsection{BrainScaleS}
  \section{Comparison of learning}
  \section{Adaptability and robustness} % How is this useful for the REF model?

\chapter{Discussion}
  \section{Validity}
  \section{REF similarity}
  \section{Future work}

\chapter{Conclusion}

\printbibliography

\appendix
\chapter{Volr} \label{appendix:volr}
  This sections presents the \gls{DSL} Volr. \index{Volr}
The main purpose of Volr is to define clear and reproducible
experiments whose semantics are retained regardless of
the runtime environment.
The specification in its current form is relatively simple, but sufficiently 
complicated for the purpose of this thesis.
It focuses solely on the topology of networks, thus
separating the network description from any generation-specific properties
of neurons or neuron populations.

The first requirement is achieved through an unambiguous syntax inspired
by the lamdba calculus \cite{Pierce2002}.
Figure \ref{fig:volr-expr} shows the BNF notation for expressions, values and types
in Volr. 
Figure \ref{fig:volr-rules} lists evaluation rules for the correct
interpretation of the expressions.

% Expression figure
\begin{figure}
  \begin{tabular}[t]{l l}
    expressions & \texttt{$e$ ::= $n$} \\
    & \begin{minipage}{0.6\textwidth}
      \begin{Verbatim}[mathescape,commandchars=\\\{\}]
    | \textbf{dense} $n\ m$
    | \textbf{let} $x = e$ \textbf{in} $e'$
    | $e\ \obar\ e'$
    | $e\ \ominus\ e'$
    | $\neg e$
      \end{Verbatim} 
      \end{minipage} \\

    & \\ % Empty space 

    values
    & \texttt{$v$ ::= $\textbf{net}\ n\ m$} \\
    
    & \\ % Empty space
    types
    & \texttt{$\tau$ ::= \textbf{int} | \textbf{net} $n\ m$} \\
  \end{tabular}

  \caption{Expressions, values and types of the Volr language.}
  \label{fig:volr-expr}
\end{figure}
\begin{figure}
\begin{prooftree}
  \AxiomC{}
  \UnaryInfC{$\Gamma \vdash n : \mathbf{int}$}
\end{prooftree}
\begin{prooftree}
  \AxiomC{}
  \UnaryInfC{$\Gamma \vdash \mathbf{stim}\ n : \mathbf{layer}\ n$}
\end{prooftree}
\begin{prooftree}
  \AxiomC{}
  \UnaryInfC{$\Gamma \vdash \mathbf{pop}\ n : \mathbf{layer}\ n$}
\end{prooftree}
\begin{prooftree}
  \AxiomC{$\Gamma (x) = \tau$}
  \UnaryInfC{$\Gamma \vdash x : \tau$}
\end{prooftree}
\begin{prooftree}
  \AxiomC{$\Gamma \vdash e_1 : \mathbf{layer}\ n$}
  \AxiomC{$\Gamma \vdash e_2 : \mathbf{layer}\ m$}
  \AxiomC{$\Gamma \vdash w : \mathbf{float}$}
  \TrinaryInfC{$\Gamma \vdash \otimes\ e_1\ e_2\ w : \mathbf{con}\ n\ m$}
\end{prooftree}
\begin{prooftree}
  \AxiomC{$\Gamma \vdash e_1 : \mathbf{layer}\ n$}
  \AxiomC{$\Gamma \vdash e_2 : \mathbf{layer}\ m$}
  \AxiomC{$\Gamma \vdash w : \mathbf{float}$}
  \TrinaryInfC{$\Gamma \vdash \ominus\ e_1\ e_2\ w : \mathbf{con}\ n\ m$}
\end{prooftree}
\begin{prooftree}
  \AxiomC{$\Gamma \vdash e : \tau$}
  \AxiomC{$\Gamma [v : \tau] \vdash e' : \tau$}
  \BinaryInfC{$\Gamma \vdash \mathbf{let}\ x = e\ in\ e' : \tau$}
\end{prooftree}

  \caption{Evaluation rules in Volr.}
  \label{fig:volr-rules}
\end{figure}



The constant expression $n$ is simply an integer that evaluates to the type 
\texttt{\textbf{int}} ($e1$). 
Similarly to the lambda calculus, the \texttt{\textbf{let}} binding binds
the string constant $x$ to the expression $e$ in an encapsulated
environment $e'$ \cite{Pierce2002}.
That constant can later be referenced in the $e'$ expression 
through the string $x$ as shown in $e2$.

The \texttt{\textbf{dense}} expression describes a network of two
populations, and is the most basic concept in the \gls{DSL}.
Notice the distinction between a neural network \textit{layer}
such as \texttt{\textbf{dense}} and a population. 
In a \texttt{\textbf{dense}} network layer, every neuron from the
first population is connected to every neuron in the second population
(\textit{densely} or all-to-all).
The two parameters $n$ and $m$ defines the number of neurons in the first and 
second layer respectively, and evaluates to the \texttt{\textbf{net}}
fundamental network type and value, as shown in $e3$. 
Considering how each neuron is a type of classifier, these numbers
illustrate the \textit{dimensionality} of the network, such that the number of
dimensions in the input is truncated (or expanded) to classify the
dimensionality of the output layer.

The $\obar$ (sequential) operator binds two networks sequentially,
such that the output layer of the first network becomes the input layer 
of the second network.
The two networks $e$ and $e'$ will share one of the neuron populations, why
the output size of the first layer is required to be equal to the input
size of the second layer ($e4$).

The $\ominus$ (parallel) operator parallelises two networks by duplicating
the input from the previous layer and merging the outputs into a single
layer ($e5$).
The input feeds into both $e$ and $e'$, such that the input dimension of
the network must be shared by the two layers ($l$). 
The output from the network is stacked such that each neuron from each
population corresponds to one output neuron ($e_{out} + e'_{out}$).
This is done to preserve the meaning of each parallel population.

Taken together these constructs can express simple neural networks and
the properties of their connections. 
Figure \ref{fig:volr-examples} shows a number of example networks
that visualises four examples of networks. 

\begin{figure}
  \ContinuedFloat*
  \begin{tabular}[t]{c c}
    \begin{minipage}{0.5\textwidth}
      \begin{Verbatim}[mathescape,commandchars=\\\{\}]
\textbf{let} s = stim 2 \textbf{in}
  \textbf{let} p = pop 2 1 \textbf{in}
    s $\otimes$ p
      \end{Verbatim}
    \end{minipage} & \begin{minipage}{0.5\textwidth}
      \includegraphics[width=\textwidth]{chapters/volr/example1.pdf}
    \end{minipage}

  \end{tabular}
  \caption{A textual and visual example of a network with a stimulus,
    a population and an implicit output.}
  \label{fig:volr-example1}
\end{figure}

\begin{figure}
  \ContinuedFloat*
  \begin{tabular}[t]{c c}
    \begin{minipage}{0.5\textwidth}
      \begin{Verbatim}[mathescape,commandchars=\\\{\}]
\textbf{let} s = stim 2 \textbf{in}
  \textbf{let} p = pop 2 1 \textbf{in}
    s $\otimes$ p
      \end{Verbatim}
    \end{minipage} & \begin{minipage}{0.5\textwidth}
      Hello
    \end{minipage}
  \end{tabular}
\label{fig:volr-examples}
\end{figure}

\FloatBarrier


\end{document}
