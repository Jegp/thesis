\newglossaryentry{agent} {
  name = agent,
  description = {An agent is a system that can act based on previous knowledge,
		 and that can \textit{learn} to adapt its actions.
	         Used interchangeably with \textit{system}}
}
\newglossaryentry{ai} {
    name = artificial intelligence,
    description = {Artificial intelligence (AI) covers the broad discipline in computer science
that is concerned with replicating intelligent behaviour in computational systems. The exact
definition is controversial for historical reasons \autocite{Nilsson2009}}
}
\newglossaryentry{ANN} {
  name = ANN,
  first = {artificial neural network},
  description = {
    Artificial neural networks is a broad term for connected units with 
    weighted edges.
    Each unit is roughly modelled over the biological neuron, in the way 
    that they can receive a number of inputs (dendrites) 
    but only provide a single output (axon).
    They also typicall use sigmoidal activation functions to calculate the 
    unit responses.
    This broad definition covers both third and second generation neural
    networks, and is generally avoided througout the thesis to avoid 
    ambiguity}
}
\newglossaryentry{API} {
  name = API,
  first = {application programming interface (API)},
  description = {A number of interaction-points for a piece of code,
		 such as a library or framework, that are available for
		 other programmers to communicate with. APIs are typically
		 documented as lists of their modular components along
		 with their purpose and usage}
}
\newglossaryentry{ARM} {
  name = {ARM},
  description = {A family of reduced instruction set computing (RISC) for 
                 processing units that are widespread in smaller and mobile
		 devices.}
}
\newglossaryentry{AST} {
  name = {AST},
  description = {An abstract syntax tree (AST) is a tree structure that 
                 represents a data model. ASTs are typically recursive.}
}
\newglossaryentry{backend} {
  name = backend,
  description = {...}
}
\newglossaryentry{BrainScaleS} {
  name = BrainScaleS,
  description = {...}
}
\newglossaryentry{computation} {
   name = computation,
   description = {Computation refers to any process (in any
substrate) that can deduce new information based on old information. In
this is manifested as computing instructions}
}
\newglossaryentry{CSA} {
  name = {connection-set algebra},
  description = {...}
}
\newglossaryentry{DARPA} {
  name = {DARPA},
  description = {The Defense Advanced Research Projects Agency is an agency
                 in the Department of Defense in the United States, who
		 have funded---and are funding---a number of emerging information
		 and military technologies.}
}
\newglossaryentry{DL} {
  name = {DL},
  first = {deep learning (DL)},
  description = {...}
}
\newglossaryentry{DNN} {
  name = {deep neural network},
  description = {Deep neural networks...}
}
\newglossaryentry{DSL} {
  name = {DSL},
  first = {domain specific language (DSL)},
  description = {A DSL is a language used to model concepts from a certain
    domain. DSLs are usually simpler than more general programming languages in
    that they contain fewer concepts and less complex syntax}
}
\newglossaryentry{FPGA} {
  name = {FPGA},
  description = {A field-programmable gate array (FPGA) is a programmable
		 integrated circuit, that can achieve high-speed computing
		 by wiring physical blocks together to perform high-speed
		 computations.}
}
\newglossaryentry{Futhark} {
   name = {Futhark},
   description = {A programming language geared towards performance in parallel environment such as
   graphics processors (GPUs). Futhark is a purely functional array language and is
   developed by HIPERFIT research center under the Department of Computer Science at the
   University of Copenhagen (DIKU)}
}
\newglossaryentry{GPU} {
  name = GPU,
  description = {Graphical processing units which are specialised processing units
                 tailored to process graphical data, typically through massive parallelism}
}
\newglossaryentry{HBP} {
  name = HBP,
  first = {Human Brain Project (HBP)},
  description = {...}
}
\newglossaryentry{NEST} {
  name = NEST,
  description = {...}
}
\newglossaryentry{NN} {
  name = {NN},
  first = {neural network (NN)},
  firstplural = {neural networks (NNs)},
  description = {A neural network refers to a circuit of neurons, artificial or biological.}
  plural = {NNs}
}
\newglossaryentry{ncc} {
   name = {NCC},
   description = {Neural patterns or condition that is minimally sufficient for a conscious
thought to occur. See \autocite{atkinson2000, Hohwy2009}}
}
\newglossaryentry{ml} {
  name = machine learning,
  description = {Machine learning is a sub-field within \gls{ai} that is concerned
    with developing systems that ``progressively improves their performance on a
    certain task'' \autocite{wiki:ml}}
}
\newglossaryentry{meme} {
name = meme,
description = {\textit{Meme} is a shortened form of the ancient Greek \textit{mimeme} meaning
`imitated thing' and was coined by Richard Dawkins. A meme refers to a idea or a
\textit{way of behaving} that can be \enquote{copied, transmitted, remembered, taught, shunned,
brandished, ridiculed, parodied, censored, hallowed} \autocite{dennett2017}}
}
\newglossaryentry{Myelin} {
  name = Myelin,
  description = {...}
}
\newglossaryentry{OpenCL} {
   name = {OpenCL},
   description = {An open standard for cross-platform parallel programming, which
   allows software to be executed on CPUs, GPUs or other processors or hardware accelerators. See \url{
   https://www.khronos.org/opencl/}}
}
\newglossaryentry{Python} {
  name = Python,
  description = {...}
}
\newglossaryentry{RAM} {
  name = RAM,
  description = {Random-access memory (RAM) is a temporary storage device
	         that allows read and write access to arbitrary locations
		 without significant delays compared to spinning disks.
		 Typically used as a cache for instructions and memory
		 from long-term storage.}
}
\newglossaryentry{REF} {
  name = REF,
  first = {Reorganisation of Elementary Functions (REF)},
  description = {A theory and model for rehabilitation in patients
  with brain lesions, developed by \cite{Mogensen2011}.
  An extension in the form of the REFGEN model was developed by
  \textcite{Mogensen2017} to account for broader aspects of
  neurocognitive organisation.}
}
\newglossaryentry{SNN} {
  name = SNN,
  first = {spiking neural network (SNN)},
  description = {A broad term for second or third generation neural
                 networks whose nodes communicates via timed pulses or
		 \textit{spikes}}
}
\newglossaryentry{vonNeumann} {
  name = {von Neumann architecture},
  description = {A computer architecture for universal computing machines that
                 relies on a processing unit, a control unit and memory for
		 storing data and instructions. Invented by John von Neumann in
		 1945.}
}
