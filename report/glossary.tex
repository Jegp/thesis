\newglossaryentry{ai} {
    name = artificial intelligence,
    description = {Artificial intelligence (AI) covers the broad discipline in computer science
that is concerned with replicating intelligent behaviour in computational systems. The exact
definition is controversial for historical reasons \autocite{Nilsson2009}}
}
\newglossaryentry{ANN} {
  name = artificial neural network,
  description = {Artificial neural networks...}
}
\newglossaryentry{BrainScaleS} {
  name = BrainScaleS,
  description = {...}
}
\newglossaryentry{computation} {
   name = computation,
   description = {Computation refers to any process (in any
substrate) that can deduce new information based on old information. In
this is manifested as computing instructions}
}
\newglossaryentry{CSA} {
  name = {connection-set algebra},
  description = {...}
}
\newglossaryentry{DNN} {
  name = {deep neural network},
  description = {Deep neural networks...}
}
\newglossaryentry{dsl} {
  name = domain specific language,
  description = {A DSL is a language used to model concepts from a certain
    domain. DSLs are usually simpler than more general programming languages in
    that they contain fewer concepts and less complex syntax}
}
\newglossaryentry{Futhark} {
   name = {Futhark},
   description = {A programming language geared towards performance in parallel environment such as
   graphics processors (GPUs). Futhark is a purely functional array language and is
   developed by HIPERFIT research center under the Department of Computer Science at the
   University of Copenhagen (DIKU)}
}
\newglossaryentry{NEST} {
  name = NEST,
  description = {...}
}
\newglossaryentry{NN} {
  name = {neural network},
  description = {...} % TODO
}
\newglossaryentry{ncc} {
   name = {NCC},
   description = {Neural patterns or condition that is minimally sufficient for a conscious
thought to occur. See \autocite{atkinson2000, Hohwy2009}}
}
\newglossaryentry{ml} {
  name = machine learning,
  description = {Machine learning is a sub-field within \gls{ai} that is concerned
    with developing systems that "progressively improves their performance on a
    certain task" \autocite{wiki:ml}}
}
\newglossaryentry{meme} {
name = meme,
description = {\textit{Meme} is a shortened form of the ancient Greek \textit{mimeme} meaning
'imitated thing' and was coined by Richard Dawkins. A meme refers to a idea or a
\textit{way of behaving} that can be \enquote{copied, transmitted, remembered, taught, shunned,
brandished, ridiculed, parodied, censored, hallowed} \autocite{dennett2017}}
}
\newglossaryentry{Myelin} {
  name = Myelin,
  description = {...}
}
\newglossaryentry{OpenCL} {
   name = {OpenCL},
   description = {An open standard for cross-platform parallel programming, which
   allows software to be executed on CPUs, GPUs or other processors or hardware accelerators. See \url{
   https://www.khronos.org/opencl/}}
}
\newglossaryentry{REF} {
  name = REF,
  description = {A model for rehabilitation in patients with brain lesions, developed
    by \cite{Mogensen2011}. An extension in the form of the REFGEN model was developed by
    \cite{Mogensen2017}}
}
\newglossaryentry{SNN} {
  name = {spiking neural network},
  description = {Spiking neural networks...}
}
